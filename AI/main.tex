\section{Начало}
\textbf{Преподаватель}: Осипов Генадий Семенович

\subsection{Литература}

\section{История ИИ}
\textbf{Искуственный интелект}~--- научное направление, возникшее 50 лет назад. В 1956 г. аналитиками американской RAND была задумана программа игры в шахматы. Имена аналитиков Ньюэл(Newel), Саймон(Simon) и Шоу(Show). Они привлекли к своему проекту известных психологов, а также Тьюринга и Клод-Шенона. В конце 1956 г. был разработан язык IPL1, предназначенный специально для задачи игры в шахматы. В 1957 г. была создана программа NSS. Она сравнивалась с шахматистом 3го разряда, но не смогла превзойти этот уровень.

\textbf{Основная идея}: уменьшение различий между текущим и целевым состоянием. Существовали правила, возможные ходы и их последствия. Для выбора следующего хода применялись эвристические правила. \textbf{Эвристика}~--- правило выбора хода без достаточного теоритического обоснования. Таким образом вычислялась оценка ходов, возможных из данной ситуации(возможно на несколько ходов) и выбирался лучший вариант. Но эвристика достигла своего потолка и потребовались новые идеи. На основе NSS были разработаны программы игры в другие игры. На основе идей были разработаны программы автоматического упрощения алгебраических выражений, автоматического доказательства теорем итд(любые задачи связанные с символьными преобразованиями). 

Программы работали с символьными преобразованиями и были разработанна теория символьных вычислений. На основе этой теории и основного типа данных~--- списка, был разработан Дж. Маккартни язык Lisp(1960г). 

В советском союзе также заинтерисовались этой идеей и в 1963г. С.Ю.Маслев предложил метод автоматического доказательства теорем вычислений предикатов. Этот метод он назвал \textbf{образтным методом}. В 1965 г. американский логик Дж. А. Робинсон предложил свой метод с аналогичным названием, который впоследствии получил название \textbf{метод резалюций}, который лег в основу языка Prolog.

На сегодняшний день есть четкое разделение на задачи исскуственного интелекта и нет. Грубо говоря~--- это поиск методов решения трудных задач, либо совокупность методов решения задач, не имеющих алгоритмического решения. 

Все это привело к появлению направления исскуственного интелекта.

Американский ученый Э. Фейгенбаум в 1975г. на международной конференции по ИИ в СССР сформулировал идею: человек повышает свою квалификацию, так как он умеет накапливать знания и умения этими знаниями пользоваться(компетенция). Программы также могут достичь высот, если будут обладать этим навыком. Это привело к созданию нового направления \textbf{Системы, основанные на знаниях}. Методы, основанные на идеях NSS выделились в отдельную область, названную \textbf{эвристическое программирование}.

Появились специальные языки \textbf{представления знаний} и средства их поддержки, системы представления знаний, методы и системы приобретения знаний. Стали исследоваться методы автоматизации(моделирования) рассуждений и методы выдвижения гипотез. Важным направлением стало \textbf{методы анализа полуструктурированной информации(текстов)}.

Таким образом \textbf{исскуственный интелект}~--- это наука, ставящая своей целью создание исскуственных устройств, способных к разумным рассуждениям, целенаправленному поведению и обучению. \textbf{Характеристика задач}: отсутствие заранее известных алгоритмов решения проблемы(например, мед. диагностика, диспечиризация энергосистемы, понимание текста итд), символьный(вербальный) характер информации. \textbf{Основные требования}: транспорентности решения задач, те прозрачности решения(пользователю должно быть понятно поведение системы в любой момент времени).

Не так давно возникло другое большое направление ИИ \textbf{когнитивное моделирование}, те моделирование аффективных и интелектуальных процессов, протекающих в человеке. В отличии от предыдущего направления здесь важно не только решение задачи, но также важно моделирование, те сам процесс решения(мотивация, причины).

Человек в отличии от животных обладает не только интелектом, но и созданием. Интелект возникает в процессе коммуникаций, а сознание в процессе социального развития(Маугли не обладал сознанием). Сознание основанно на системе знаков, те сложных связей смыслов, образов, понятий итд.

В 1999 г. американцами был запущен космический корабль "<Deep space-1">, который управляется 6 интелектуальными системами принятия решения.

\section{Представление знаний}
Первая задача ИИ~--- это построение такого элемента, как \textbf{база знаний}. База знаний должна содержать знания о предметной области. Но для этого требуется спец. средства: \textbf{языки описания знаний}. Таких систем несколько:
\begin{itemize}
	\item Системы правил.
	\item Семантические сети.
	\item Системы фреймов.
	\item Комбинация.
\end{itemize}

Для начала нужно изучить логические аспектц представления знаний.
\subsection{Логические аспекты представления знаний}
Языки представления знаний обладают гораздо большей логичностью, чем системы вычисления предикатов 1-го порядка, основанные на математике. Это связано с тем, что в случае изучения поведения системы, появления новых знаний ранг формулы может поменятся, что не учитывается в системах вычисления предикатов. Например, формула "телефон лежит на столе" становится ложной, когда мы берем телефон в руку. Это невозможно описать в системах вычисления предикатов.

Проблемы естественного языка:
\begin{itemize}
	\item Многозначность языка (полисемия).
	\item Противоречивость языка.
	\item Возможность "<кантрабандного протаскивания информации">. Использование понятий не введенных заранее.
\end{itemize}

Поэтому на этапе создания базы знаний необходимо пользоваться исскуственными языками. Далее можно применять для исследования естесственных языков.

Избавившись от проблем в естсесственном языке мы получаем исскуственный язык, добавив некоторые правила получаем формальные языки.

\subsection{Язык вычисления предикатов 1-го порядка(ЯИППП)}
Любой язык~--- это цепочки символов, пока не заданна интерпретация. Так в математике $x$ всего лишь переменная, пока не заданно ее значение, например, высота дома или сила тока.

Основной конструкцией языка является формула, котороая строится на основе алфавита. Поэтому сначала необходимо задать алфавит. Мы возьмем счетное число малых букв конца латинского алфавита возможно с индексами(x,y,z,...) для обозначения переменных языка. Возьмем счетное число малых букв начала латинского алфавита для обозначения констант(a,b,c,...). Возомем счетное число букв середины латинского для обозначения функциональных символов(f, g, ...). Каждому символу присвоим число n, которое называется его местностью или арностью. Счетное число больших букв для обозначения предикатов(P,Q, ...).

Обозначим операторы $\rightarrow$~--- логическая связь, логическое отрицание($\neq$) и квантор всеобщности, вычисляется для всех.

Этот язык является минимальным, но полным.

\paragraph{Терма}. Начнем с понятия \textbf{термы}.
\begin{enumerate}
	\item буква переменной есть терма.
	\item буква для обозначения константы есть терма.
	\item если f~--- l-местный функциональный символ, а $t_1,t_2,\ldots,t_n$~--- термы, то $f(t_1,t_2,\ldots,t_n)$~--- терм.
	\item если P~--- l-местный предикатный символ,  а $t_1,t_2,\ldots,t_n$~--- термы, то $P(t_1,t_2,\ldots,t_n)$~--- атомарная формула.
	\item атомарная формула есть формула.
	\item Если $F_1,F_2$~--- формулы, то $F_1\rightarrow{}F_2$~--- формула и $\neg{}F_1,\neg{}F_2$~--- формулы.
	\item Если $F$~--- формула, то для $(\forall{}x), F$~--- формула.
	\item Всякое выражение явялется формулой, только если удовлетворяет усовиям 1-7.
\end{enumerate}

Примеры формул:
\begin{align*}
\forall{]x\exists{}y\text{Ф}(x,y) \\
F_1(x)\rightarrow{}F_2(x)
\end{align*}

Выразим дезьюнкцию:$A\rightarrow{}B\equiv{}\neq{}A\or{}B$.

Если квантор предшествует некоторой переменной и эта переменная находится в области действия квантора, то эта переменная называется \textbf{связаной}, иначе \textbf{свободной}. Например, $\forall{}x(F(x)\rightarrow{}\text{Ф}(y))$, где $x$~--- связанная переменная, а $y$~--- свободная. Атомарные формулы без свободных переменных называют \textbf{фактором}.

Определим понятие осмысленности. Например, $\forall{}x\exists{}y, >(y,x)$. Осмысленность зависит от областей изменения $x,y$, те их \textbf{интерпретации}. Если это натуральные числа, то тут задана аксеома о бесконечности натурального ряда и она осмысленна. Если задать $x,y$, как высоты домов, то формула неосмысленна и ложна.


Существуют формулы, которые не зависят от интерпретации, их называют \textbf{классические логические аксеомы}. 
\begin{itemize}
	\item $(A\rightarrow{}(B\rightarrow{}A))$.
	\item $(A\rightarrow{}(B\rightarrow{}C))\rightarrow((A\rightarrow{}B)\rightarrow(A\rightarrow{}C))$.
	\item $((A\rightarrow{}B)\rightarrow(\neq{}B\rightarrow{}\neq{}A))$.
\end{itemize}

Предикат в котором отсутствую свободные переменные называют \textbf{высказыванием}, в свою очередь сам символ предиката становится переменной. 

Существует 2 вида переменных: индивидный и пропозициональный. Пропозициональный вид переменной~--- переменная принимает значение 1 или 0.

Язык превращается в исчисление, когда появляются механизмы, позволяющие выводить новые формулы из старых. Такихе механизмы называются \textbf{правилами вывода}. Например, в дифференциальном исчислении существует алфавит:горизонтальная черта, две буквы d(сверху и снизу). Существуют правила записи формул, существуют аксеомы(таблица простейших производных) и правила вывода(производная от произведения итд).

Необходимо определить аксеомы для индивидных переменных:
\begin{itemize}
	\item Аксеома генерализации. $(\forall{}x)((A\rightarrow{}B)\rightarrow(A\rightarrow(\forall{}x)B))$.
	\item Аксеома спцеификации. $A(t)\rightarrow{}A(x)$, где t~--- терм, а x~--- не содержится в t и является свободной пременнной.
\end{itemize}

\subsubsection{Правила вывода}
\begin{enumerate}
	\item \textbf{Правило отделения}. Если выводимо A и выводимо $A\rightarrow{}B$, то выводимо B.
	\item \textbf{Правило подстановки}. В любую аксеому на место любой пропозициональной переменной можно подставить любое высказывание или выражение, предварительно переименовав пропозициональные переменные подставляемого предложения тока, чтобы они не совпадали с переменной исходной формулы.
	\item \textbf{Правило обобщения}. Если выводимо A, то выводима $\forall{}xA$, где $x$~--- свободная переменная в A.
\end{enumerate}

Эти аксеомы действительны для любой области, так как определяют не правила какой-то конкретной области, а описывают законы нашего мышления.

\subsubsection{Алгебраические системы}
Пусть $M$~--- некоторое множество, а $F={F_1,F_2,\ldots,F_n}$~--- семейство функций $F_i:M^n\rightarrow{}M$. А $R\subseteq{}M^n$~--- семейство отношений. Тогда $A=<M,F,R>$~--- \textbf{алгебраическая система}. Здесь $F$~--- n-мерная функция, результатом которой является одно едиснтвенное значение из $M$. Здесь $R$~--- n-местное отношение, которое является подмножеством множества всех кортежей.

Определим $M^n$. Пусть заданны 2 множества $M_1,M_2$, тогда $M_1\times{}M_2$~--- называют декартовым произведением. Те это множество всех пар элементов из $M_1$ и $M_2$. Если записать $M^2$~--- это множество всех возможных пар элементов из $M$. Если количество множеств больше 2х, то результатом произведения будет являться множество всех упорядоченных наборов элементов из этих множеств(кортежей).

\subsubsection{Семантика или исчисление предикатов 1-го порядка}
Пусть заданно некоторое исчисление(исчисление предикатов 1-го порядка) и задан некоторый универсум $M$(большое множество, бесконечное, множество всех элементов). Задаем некоторое отображение "I", которое каждому некоторому константному символу $a$ из исчисления предикатов ставит в соответствие некоторый элемент $m$ из универсума. Всякому n-местному функциональному символу $f$ ставит в соответствие местную функцию $G:M^n\rightarrow{}M$. Всякому n-местному предикату $P$ ставится в соответствие местное отношение $R\\subseteq{}M^n$. Тогда алгебраическая система $S=<M,G,R>$~-- \textbf{интерпретация} или модель формальной системы. $I(a)=m, I(P)=R, I(f)=G$
Дадим определение истиности. 
\begin{itemize}
\item Пусть $P$~--- n-мерный предикатный символ $x_1,x_2,\ldots,x_n$~--- свободные переменные. Будм говорить, что атомарная формула $P(x_1,x_2,\ldots,x_n)$ выполняется в модели $\Omega$, если существует подстановка $a_1/x_1,a_2/x_2,\ldots,a_n/x_n$, где $a_1,a_2,\ldots,a_n$~--- константы, такие что $(I(a_1),I(a_2),\ldots,I(a_n)\in{}I(P)$. Где $\Omega=<M, I>$. Если формула выполняется на любой подстановке, то она называется \textbf{истиной модели}.

\item Формула $\forall{}x_1,\forall{}x_2,\ldots,\forall{}x_nP(x_1,x_2,\ldots,x_n)$ истина в модели $\Omega$, если $P(x_1,x_2,\ldots,x_n)$~--- истинна в этом отношении.
\end{itemize}

\textbf{Теорема Геделя(о полноте)}. Произвольное предложение из языка L выводимо в исчислении предикатов первого порядка тогда и только тогда, когда оно истино.

\subsection{Системы правил}
Системы правил возникли на заре развития ИИ, когда Ньювел и Саймон стали заниматься ИИ и ходы в их шахматной программе были заданны правилами. В дальнейшем эта идея легла в основу программы "<Универсальный решатель задач"> GTS. На основе этой программы был созданн первый робот StRIPS в Институте Стенфорда.

Правилом называют упорядоченную тройку множетсв $\Pi=<C,A,D>$, где $C$~--- условие правила, $A$~--- множество добовляемых правилом фактов, $D$~--- множество удаляемых правилом фактов. Устройство $C,A,D$ одинаково~--- это множества атомарных формул вычисления предикатов первого порядка.

В каждой из систем выделяются 3 компоненты:
\begin{itemize}
	\item Множество правил.
	\item Рабочая память(база знаний).
	\item Набор стратегий(стратегия управления).
\end{itemize}

Для начала введем понятие \textbf{понятие выполнимости правила}:Будем считать, что условие правила выполнено, если в текущем состоянии рабочей памяти выполянется каждая из атомарных формул условия.

\subsubsection{Стратегии управления}
Напишем простейшую стратегию:
\begin{enumerate}
	\item Выбрать некоторое правило из множества правил.
	\item Проверить выполнимость правила в текущем состоянии выполнения правил.
	\item Если условие выполянется, то применить его.
	\item Перейти к пункту 1.
\end{enumerate}
Тут присутствует слово "<некоторое">, оно означает выбор правила по какому-то критерию. В простейшем случае правила могут быть отсортированны в лексико-графическом порядке.

Проверка выполнимости определена выше, те проверить выполнимость каждой атомарной функции. 

Выполнение правила значит добавление в рабочую память фактов множества A и удаление предикатов множества D.

Система развивается, рабочая память изменяется и работа завершается либо когда достигнута цель работы, либо если нет возможности применить ни одного правила.

В случае, если необходимо выбрать не просто стратегию, а наилучшую стратегию, то необходимо перебрать все возможные варианты.

\paragraph{Стратегия 2}
\begin{enumerate}
	\item Выбрать некоторое правило из множества правил.
	\item Проверить выполнимость условия правила, если условие правило выполнено, то поместить его в конфликтное множество правил и перейти к пункту 1.
	\item Если множество конфликтных правил исчерпано, то выбрать наилучшее правило из конфликтного множества правил и перейти к пункту 1.
\end{enumerate}

Тут вступает в силу эвристика для выбора наилучшего правило. Существует универсальное правило:каждый раз необходимо выбирать наиболее конкретное правило. \textbf{Наиболее конкретное правило}~--- это правило, которое наиболее полно описывает текущую ситуацию. Наиболее полно, значит содержит наибольшее количество предикатных символов и конкретизированных переменных. 

Существует область комутативных задач, где выбор правила не имеет значения, те выбор неудачного правила не препятствует решению задачи, а только отодвигает его. Достаточное условие $D=\emptyset$.

Особенность этого в том, что данное решение не предусматривает наличие какого-либо алгоритма. Мы объявляем правила, известные факты и запускаем модули. Получив решение мы можем построить алгоритм, но до этого момента мы не можем найти алгоритм.

\paragraph{Пример}. Имеется некоторый робот, который умеет строить башни из кубиков. Нужно написать систему, основанную на правилах, управляющую деятельностью робота. Для этого нужно описать предметную область. Объявим предикатные символы:
\begin{itemize}
	\item $On$~--- двуместный предикатный символ "находится на",
	\item $Em$~--- одноместный предикатный символ "не находится под кубиком",
	\item $Er$~--- одноместный предикатный символ "находится на земле".
\end{itemize}
Необходимо описать правила: брать кубик можно, если он на земле;ставить кубик можно только если на кубике не стоит другой кубик. При этом первый кубик более не стоит на земле, а второй не свободен более сверху.

Правило:
\begin{align*}
	\Pi_1=<C_1,A_1,D_1> \\
	C_1={Em(y),Er(x),Er(y)} \\
	A_1={On(x,y)} \\
	D_1={Em(y), Er(x)}
\end{align*}

При этом мы не сможем достичь поставленной цели, поэтому объявим другое правило:
\begin{align*}
	\Pi_2=<C_2,A_2,D_2> \\
	C_2={Em(x),Er(x),Em(y),On(y,z)} \\
	A_1={On(x,y)} \\
	D_1={Em(y), Er(x)}
\end{align*}

Таким образом данный спсоб можно применять, если предметная область \textbf{хорошо структурированна}. Те известен список всех элементов предметной области, известен полный набор свойств и признаков объектов этой области и известна система связей между объектами области. 

Однако существует огромное количество плохо структурированных областей. В основном это естественные и гуманитарные науки. Для таких областей создаются другие способы представления знаний.

\section{Семантические сети}
Появившись изначально для хорошо структурированных систем, но могут применяться для некоторого множества плохо структурированных задач. Семейство графов над общим множеством вершин, вместе с присоедененными процедурами. Процедуры нужны для того, чтобы интерпретировать ребра семантических сетей.

Роб. Ковальский(1968 г.) ввел понятие \textbf{расширенных семантических сетей}. Такие сети могут включать различные ребра, которые несут различный смысл и различные вершины. Закрашенные~--- константы, незакрашенные~--- переменные. Эти сети полностью соответствуют исчислению предикатов 1-го порядка.

\subsubsection{Неоднородные семантические сети}
Эти сети применяются для описания плохо структурированных предметных областей. Неоднородной семантической сетью называют систему $H=D,N,R,F$, где $D$~-- семейство непустых множеств, $N$~--- выделенное подмножество множества слов конечной длинны, $R$~--- семейство бинарных отношений, а $F$~-- семейсвто функций, каждой из которых преписан некоторый тип.

Рассмотрим отображение, которое каждому слову из этого множества ставит в соответствие подмножество из декартового произведения типа $\tau$ из множества $D$. Выделим каждому типу декартового произведения из $D$ подмножество $e$, которое назовем \textbf{экзистенционалом}, а $E$~--- множество всех экзистенционалов. Если определить отношения на $E^2$, так,что все элементы эксистенционалов соответствуют указанных отношений. Получим новую сеть, называемую \textbf{эксистенциональной}.

Особенность подхода в том, что все отношения~--- бинарные. Необходимо определить отношения $R$. Выделим основные свойства отношений:
\begin{itemize}
	\item Отношение R транзитивно, если $\forall{}a\forall{}b\forall{}c(a,b)\in{}R\AND(b,c)\in{}R\rightarrow(a,c)\in{}R$.
	\item Симметричность.
	\item Рефлексивность.
\end{itemize}
Транзитивное, симметричное и рефлексивное отношение называют отношение \textbf{эквивалентности}.

Транзитивное, рефлексивное и антисимметричное отношение называют отношением \textbf{частичного порядка}.