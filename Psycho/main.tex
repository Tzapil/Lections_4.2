\section{Введение}
\subsection{Преподаватель}
\textbf{Зиновьева Татьяна Германовна}

\subsection{Список литературы}
\begin{enumerate}
	\item Столяренко Л. Д. "<Основы психологии">. Ростов-на-Дону, 2009г.
	\item Немов Р. С. "<Психология в 3х томах">. Москва, 2012г.
	\item Пидкасистый П. И. "<Педогогика. Учебное пособие для бакалавров">. Москва, 2012г.
	\item Подласый И. П. "<Педагогика для бакалавров">. Москва, 2012г.
\end{enumerate}

\section{Предмет психологии}
\subsection{Этапы становления психологического знания}
Понятие психология происходит от слов \textbf{psyhe}~--- душа и \textbf{logos}~--- знание (наука о душе).

\textbf{Первый этап} становления психологического знания в эпоху античности. Складываются различные подходы к пониманию психики. Основными считают идеалистический подход и материалистический. Психикой наделялись и неодушевленные объекты. Был поставлен вопрос о различии людей в темпераменте (Гиппократ, Аристотель), проблема взаимодействия рационального и эмоционального в поведении человека (Платон).

\textbf{Первый психологический труд}~--- это трактат Аристотеля "<О душе">. В нем исследуются 4 психические функции:
\begin{itemize}
	\item двигательная,
	\item питательная,
	\item чувствующая,
	\item разумная.
\end{itemize}

\textbf{Второй этап} становления в XVII в. связан с исследованием знания и мышления. Важными учеными являются:
\begin{enumerate}
	\item Рене Декарт. Заложил основы изучения сознания. Психика функционирует только на сознательном уровне, предлогает первый метод исследования психических явлений~--- \textbf{самонаблюдение}.
	\item Лейбниц. Ввел понятие безсознательной психической деятельности. 
	\item Дж. Локк. Считается родоначальником эмперической психологии (обобщение практического материала на основе трудов ученых, путешественников, церковнослужителей). Изучал механизмы функционирования сознания. Основатель идеи ассоциаций (ассоционизм).
\end{enumerate}

\textbf{Третий этап} приходится на XIX в. Становление психологии, как самостоятельной науки, что было обусловлено развитием эксперементальной психологии. В 1879 г. немецкий ученый В. Вундт открывает первую эксперементальную психологическую лабораторию. В России первая лаборатория открыта в 1885 г. В. Бехтеревым (вторая в Европе).

Таким образом к концу XIX в. сложился ряд направлений психологического исследования. Основынми считаются:
\begin{itemize}
	\item изучение сознания,
	\item изучение поведения,
	\item изучение безсознательного.
\end{itemize}

\subsection{Объект и предмет психологии}
\textbf{Объект психологии}~--- это психические явления и поведение человека. Психические явления представлены тремя группами.
\begin{enumerate}
	\item Психические процессы.
		\begin{enumerate}
			\item Познавательные.
			\item Эмоционально-волевые.
		\end{enumerate}
	\item Психические состояния.
	\item Психические свойства личности.
		\begin{enumerate}
			\item Темперамент.
			\item Характер.
			\item Способности.
			\item Направленность.
		\end{enumerate}
\end{enumerate}

\textbf{Предмет психологии}~--- это факты, категории и закономерности (устойчивые взаимосвязи между псих. явлениями), отражающие функционирование психики.

\textbf{Психика}~--- системное качество и результат деятельности мозга. Необходимыми условиями развития психики являются общение с другими людьми и полноценное, здоровое функционирование головного мозга.

\subsubsection{Стуркутра психологии}
\begin{itemize}
	\item \textbf{Общая психология}. Изучает общие особенности психических явлений, закономерности функционирования психики, методы психологического исследования и другие общие вопросы.
	\item Психология развития.
		\begin{itemize}
			\item Возрастная психолгия. Особенности психического развития на различных возрастных этапах.
			\item Патопсихология. Отклонения психического развития.
		\end{itemize}
	\item Социальная психология. Изучает влияние социальных факторов на психику человека.
	\item Прикладная психология. Изучает практические, жизненные проблемы (например, педогагическая психология).
\end{itemize}

\subsubsection{Методы психологии}
\begin{itemize}
	\item \textbf{Наблюдение}.
		\begin{itemize}
			\item Внешнее. Наблюдение со стороны.
			\item Включенное. Наблюдение, как члена группы и тд.
			\item Самонаблюдение. Анализ своего сознания.
		\end{itemize}
	\item \textbf{Эксперимент}.
		\begin{itemize}
			\item Лабораторный. Использование спец оборудования, помещения, условий и тд.
			\item Естественный. В естественной среде.
		\end{itemize}
	\item \textbf{Опрос}.
		\begin{itemize}
			\item Интервью.
			\item Анкетирование.
		\end{itemize}
	\item \textbf{Тестирование}. Это чисто психологический метод. Тестирование (англ. испытание, проверка, проба)~--- это ограниченое во времени испытание, предназначеное для выявления индивидуальных психологических различий. Целью тестирования является выявление особенностей конкретной личности.
	
	Существуют совершенно различные методы тестирования, например, скорость реакции, анкеты, реакция на изображения, прямые вопросы и тд. Это связано с тем, что не существует абсолютно универсальных тестов. Различные тесты направлены на выявление различных черт личности, существует развития и изменения и тд. Не все тесты научно-обоснованны.
	
	В современном мире популярно проводить псих. тесты при приеме на работу, в школу, институт и тд.
	
	По содержанию различают:
		\begin{itemize}
			\item тесты интелекта (например, IQ),
			\item тесты способностей (тесты на изучение способностей в определенной сфере, например, математические),
			\item личностные тесты (качества личности, характер и тд.),
			\item тесты достижений (оценки обучаемости и профессиональных знаний).
		\end{itemize}
	\item \textbf{Изучение документов}. Изучают личные документы и официальные.
	
	Изучение личных документов дают много информации о личности (например, письма и дневники).
\end{itemize}

\subsection{Современные психологические направления}
\subsubsection{Психоанализ}
Это направление возникло в конце XIX в. основоположником считается \textbf{З. Фрейд}. Сфера изучения~--- безсознательное. Используются такие методы, как гипноз, метод свободных ассоциаций, толкование сноведений, метод ошибочных действий и тд.

Структура сознания по Фрейду:
\begin{enumerate}
	\item Оно~--- безсознательное,
	\item Я~--- сознательное,
	\item Сверх Я~--- внутрений цензор.
\end{enumerate}

В начале своей карьеры Фрейд использует гипноз, но впоследствии от него отказывается. Он разрабатывает метод свободных ассоциаций (например, показ картинок, просьба сказать первое, что пришло в голову при слове и тд).

Кроме Фрейда большой вклад в развитие данного направления внес его ученик \textbf{К. Юнг}. Юнг выделил 2 вида безсознательного:
\begin{itemize}
	\item индивидуальное,
	\item коллективное.
\end{itemize}

Коллективное безсознательное состоит из архитипов.

Также представителем является \textbf{А. Адлер}. Исследовал комплексы неполноценности. Ввел понятие мотивации. Изучение людей, выросших в неблагоприятной среде и имеющих различные комплексы привело его к выводу, что такие люди часто достигают высот в следствии стремления доказать свое превосходство (например, Наполеон).

\subsubsection{Бихевиоризм}
Направление возникает в начале XX в. в США, основоположником считается Дж. Уотсон. \textbf{Объект изучения}~--- это поведение, поведение, понимаемое, как реакция организма на стимулы внешней среды в процессе адаптации ($S\rightarrow{}R$). Используется экспериментальный метод исследования. Отечественным прообразом стала "<рефлексология">.

Э. Толмен дополняет теорию, добавляя такое понятие, как "<медиатор"> ($S\rightarrow{}I\rightarrow{}R$).

\subsubsection{Гумманистическая психология}
50-60е гг. XX в. Наиболее известные представители: А. Маслоу, К. Роджерс, В. Франкл. 

Человек исследуется, как в высшей степени сознательное существо, стремящееся к саморазвитию и самореализации. Отрицается возможность проведения эксперемента в психологии.

А. Маслоу считал, что потребности человека представляют определенную иерархическую структуру:
\begin{enumerate}
	\item физиологические,
	\item безопасность (как физическая, так и социальная)
	\item общение,
	\item самоутверждение,
	\item самореализация.
\end{enumerate}
Более высокие потребности возникают, когда нижележащие удовлетворены. Разные люди удовлетворены различными уровнями пирамиды, например до уровня самореализации доходит только 20\% всего населения.

К. Роджерс работал психотерапевтом и имел богатый практический опыт. Считал, что человек по природе добр и все проявления жестокости, агрессии~--- это отклонения от нормы. Он выделил в структуре личности 2 составляющие:
\begin{itemize}
	\item Я-идеальное (желания, мечты и то, что человек хочет достичь),
	\item Я-реальное (реальное состояние дел).
\end{itemize}
Я-реальное должно максимально стремиться к достижению Я-идеальное.

В. Франкл разработал учение "<Логотерапия">. Имел богатый практический опыт, много работал в концлагерях. Считал, что если у человека есть "<смысл жизни">, цель, то он может преодолеть огромные трудности для достижения этой цели. Смысл в том, чтобы помочь человеку найти именно его цель.

\subsubsection{Когнитивная психология}
Направление возникает в середине XX в. Основные деятели: Ж. Пиаже, Л. Колберг.

Человек рассматривается, как анализирующий субъект, способный воспринимать информацию, принимать решения, разрешать проблемы. Поступок человека включает 3 компонента:
\begin{itemize}
	\item само действие,
	\item мысли,
	\item чувства, испытываемые при выполнение действия.
\end{itemize}
Внешне похожие действия могут быть разными, так как мысли и чувства человека были иными.

Ж. Пиаже исследовал процессы мышления на различных возрастных этапах. Он пологал, что процесс развития мышления идет индивидуально, длительность каждого этапа зависит от конкретного человека. В среднем к 14-и годам у человека формируется "<понятийное мышления">, те оперирование понятиями и логическими выводами.

Л. Колберг изучал нравственную структуру личности.

\subsubsection{Отечественная психология}
Становление относится к концу XIX в. У истоков стояли: И. Сечинов. И. Павлов, Г. Челпанов и В. Бехтерев. Их часто называют "<объективными психологаим">, а направление \textbf{Объективная психология}. Широко использовали эксперименты в исследованиях.

В экспериментах над мозгом Сечинов открыл тн "<тормозные центры">, что позволило ввести в психологию понятие \textbf{торможения}. Наиболее значительная работа Сечинова "<Психологические этюды"> в 1873 г.

Бехтерев развивал энергитическую концепцию, которая объясняла взаимодействие психики и физиологии. Исследовал также природу и динамику группового и массового поведения, уделял внимание проблемам общения.

Павлов разработал теорию \textbf{высшей нервной деятельности}. Он продолжал работы Бехтерева и Сечинова. Говорил, что скорость нервной деятельности у разных людей различна (выделил 4 типа, которые согласовал с 4мя основными типами людей:холерик, сангвиник, флегматик, меланхолик; например холерик имеет крепкую нервную систему, но возбуждение преобладает над торможением, а сангвиник более устойчив).

Г. Челпанов в 1912 г. создал психологический институт.

После этого в 30х годах началось идеологическое преследование и многие ведущие психологи уехали на запад.

После в 60х гг. во время Хрущевской оттепели снова начинает развиваться. Деятели: Л. Выготский, С. Рубенштейн, А. Лурия, Г. Андреева А. Леонтьев.

Выготсикй изучал проблемы мышления и речи, внес вклад в исследование психопаталогий.

Рубенштейн основал факультет психологии в МГУ в 1942 г. Основал \textbf{деятельностный подход} к психологии.

Лурия развивал теорию культурно-исторического функционирования психики, также исследовал эмоциональные процессы.

В настоящее время хорошо развиваются психофизиология и другие практические направления психологии (например теории мотивации, психология организаций и тд). Хорошо развиваются новые направления психологии (например изучение влияния СМИ и интернет, исследование игромании и тд). Также развивается психодиагностика и компьютерная психодиагностика.

\section{Психология позновательных процессов}
\subsection{Ощущение, восприятие, внимание, как позновательные психические процессы}
\subsubsection{Ощущение}
\textbf{Ощущение}~--- это отражение отдельных свойств объекта при его непосредственном воздействии на органы чувств. Ощущения бывают зрительные, слуховые, обонятельные, осязательные, вкусовые, температурные, болевые, кинестетические (ощущения движения), интероцептивные (ощущения внутренего состояния организма). Анатомо-физиологический аппарат предназначеный для приема ощущений называется \textbf{анализатором}.

Ощущения характеризуются верхним и нижним порогом (максимальная и минимальная величина ощущения, которое мы можем ощутить), дифференциальным порогом (шаг с которым мы можем отличить ощущения, например два оттенка красного) и латентным периодом (время, которое проходит между воздействием раздражителя и начала ощущений).

Ощущение~--- это самый простой из позновательных процессов, с которого начинается процесс познания мира.

Ощущения бывают сознательные и безсознательные (например, ощущение равновесия).

\subsubsection{Восприятие}
\textbf{Восприятие}~--- это целостный образ объекта или явления при его непосредственном воздействии на органы чувств.

Виды:
\begin{itemize}
	\item восприятие времени,
	\item пространства,
	\item движения,
	\item отношений,
	\item объекта.
\end{itemize}

Восприятие характеризуется:
\begin{itemize}
	\item постоянством (существенные характеристики объекта быстро не меняются, поэтому мы воспринимаем объект постоянно),
	\item структурностью,
	\item избирательностью (объекты воспринимаются достаточно субъективно, избирательно по отношению к существенным свойствам объекта),
	\item зависимостью от прошлого опыта человека.
\end{itemize}

\subsubsection{Винимание}
\textbf{Внимание}~--- это сосредоточение на одном или нескольких объектах при одновременном отвлечении от других.

Внимание может быть непроизвольным, когда воздействует неожиданный контрастный или новый раздражитель. Произвольное внимание связано с сознательно поставленной целью, утомляет через 15 минут. Послепроизвольное внимание возникает при появлении интереса к деятельности.

Объем внимания у ребенка равен 2-3 объекта, а у взрослого человека 4-6 объектов.

Внимание характеризуется:
\begin{itemize}
	\item концентрацией,
	\item распрелением,
	\item переключением.
\end{itemize}

\subsection{Память в системе позновательной деятельности}
\textbf{Память}~--- это позновательный процесс, заключающийся в запоминании, хранении и последующем воспроизведении информации. Это вознейший позновательный процесс, который позволяет сохранять иформацию, знания, адаптироваться к окружающей среде.

По способу запоминания различают:
\begin{itemize}
	\item непроизвольную память (информация запоминается сама собой без спец. заучивания),
	\item произвольная память (целенаправленое заучивание информации, зависит от цели запоминания и приема запоминания).
\end{itemize}

Первоначально информация попадает в кратковременную память, где хранится 5-7 минут. Объем кратковременной памяти равен $7\pm2$ еденицы информации. Далее информация либо забывается, либо переходит в промежуточную память, где она хранится в течении дня. Во время сна информация из промежуточной памяти обобщается, оценивается и пересылается в структуры долговременной памяти.

Долговременная память бывает  с сознательным доступом и закрытая.

Оперативная память обслуживает конкретные виды деятельности, включает информацию как из кратковременной памяти, так и из долговременной.

\subsubsection{способы воспроизведения информации}
\begin{itemize}
	\item Припоминание.
	\item Узнавание (при повторном восприятии объекта).
	\item Реминисценция (озарение).
\end{itemize}

\subsection{Мышление и воображение}
\subsubsection{Мышление}
\textbf{Мышление}~--- это позновательный процесс, заключающийся в установлении взаимосвязей и отношений между позноваемыми объектами.

Классификации:
\begin{itemize}
	\item В зависимости от степени осознаности мыслительной деятельности.
		\begin{itemize}
			\item Аналитическое мышление.
			\item Интуитивное мышление.
		\end{itemize}
	\item По степени новизны результатов мыслительной деятельности.
		\begin{itemize}
			\item Репродуктивное мышление (воспроизводство старых образцов, схем, алгоритмов).
			\item Творческое мышление.
		\end{itemize}
	\item По способу осуществления.
		\begin{itemize}
			\item Наглядно-действенное.
			\item Наглядно-образное.
			\item Словесно-логическое.
		\end{itemize}
\end{itemize}

\textbf{Понятийное мышление} характерно для взрослых людей и связано с использованием понятий и логических операций.

\subsubsection{Воображение}
\textbf{Воображение}~--- это процесс создания образов. Оно может быть пассивным (сноведения, грезы, мечты) и активным (например в процессе творчества). 

К формам активного воображения можно отнести:
\begin{itemize}
	\item гиперболизацию (преувеличение некоторых свойств),
	\item схемотизация,
	\item типизация,
	\item агглютинация (соединение несоеденимых элементов).
\end{itemize}

\section{Сознание, как высшая ступень развития психики}
\subsection{Понятие, функции, процесс развития сознания}
\textbf{Сознание}~--- это высшая форма психического отражения, в ходе которого у человека формируется внутреняя модель внешнего мира.

Функции сознания:
\begin{itemize}
	\item позновательная,
	\item эмоционально-оценочная,
	\item регулятивная,
	\item креативная (творческая),
	\item рефлексивная (способность сознания сосредотачиваться на самом себе).
\end{itemize}

Процесс развития сознания можно рассматривать по ряду направлений.
\begin{itemize}
	\item Рефлексивное направление~--- это переход от познания внешнего мира к познанию самого себя, в результате у человека формируется самосознание.
	
	Самосознание имеет 4х компонентную структуру:
		\begin{itemize}
			\item осознание себя, как активного субъекта деятельности,
			\item неповторимость,
			\item социально нравственная самооценка,
			\item самооценка своих психических свойств.
		\end{itemize}
	\item Понятийное направление~--- процессы развития сознания тесно связано с процессами развития мышления и речи.
	
	По мере развития знаний об окружающем мире у человека формируются понятия, которые используются для обозночения групп предметов с исходными существенными признаками. Формируется словесно-логическое мышление.
	\item Историческое направление~--- каждая историческая эпоха накладывает отпечаток на сознание ее современников, изменение объективных условий жизнедеятельности людей всегда приводит к изменениям в их сознании.
\end{itemize}

\subsection{Взаимодействие сознания и безсознательного}
Безсознательное представлено во многих психических явлениях. В зависимости от взаимодействия с сознанием выделяют следующие уровни безсознательных явлений:
\begin{itemize}
	\item область предсознания (например, безсознательные ощущения, неосознаваемые образы восприятия, интуиция, установки),
	\item безсознательные явления. которые ранее осознавались, но в силу выработки автоматизма ушли в безсознательное (например, двигательные навыки, оперирование предметами, привычки, профессиональные навыки),
	\item личностное безсознательное (то, что вытеснено из сферы сознания под воздействием моральных запретов; может проявляться в сноведениях, грезах, мечтах, в непроизвольном путнии/забывании имен и событий, с которыми у человека связаны неприятные воспоминания, ошибочные действия; эта область наиболее важна для изучения по мнению психоаналитиков).
\end{itemize}

Безсознательное регулирует повседневные рутинные действия, освобождая от этих функций сознание.

\subsection{Эмоции и воля, как характеристики сознания}
\textbf{Эмоциональные процессы}~--- это переживания, отражающие личную оценку и значимость события для жизнедеятельности человека. Многообразные проявления эмоциональной жизни делятся на:
\begin{itemize}
	\item аффекты~--- это бурныем, относительно кратковременные, эмоциональные реакции человека на только что произошедшее событие,
	\item эмоции~--- это состояния, которые могут носить опережающий характер; более длительно по времени, чем аффект,
	\item чувства~--- это устойчивое отношение к кому-либо или чему-либо,
	\item настроение~--- это эмоциональный процесс, влияющий на эффективность деятельности человека,
	\item стресс~--- это состояние поведенческого и душевного расстройства, связанное с невозможностью человека целесообразно и разумно действовать (изучал Г. Селье, выделил дистресс~--- отрицательно влияющий на человека и аутстресс~--- мобилизация ресурсов для решения проблем). При длительном стрессе возникает фаза усталости и исчерпания ресурсов.
\end{itemize}

Для создания оптимального эмоц. состояния требуется правильно оценивать события, собирать объективную разноплановую информацию по проблеме, продумывать запасные варианты действий в сложных ситуациях.

\textbf{Воля}~--- сознательное регулирование человеком своего поведения, связаное с преодолением внешних и внутрених препятствий.

\section{Психология личности}
\subsection{Психические свойства человека}
\subsubsection{Психологическая струкутра личности}
\begin{table}
\begin{tabular}{|c|c|}
	Биологически обусловленая подструктура & Биологически обусловленая подструктура\\
	\hline
	Темперамент & Социальный опыт\\
	Тип нервной системы & Характер\\
	Конституция тела & Направленность личности\\
	Пол & \\
	Возрастные особенности & \\
	Патологические особенности & \\
	\hline
	\multicolumn{2}{|c|}{Особенности психических процессов,}\\
	\multicolumn{2}{|c|}{Способности}\\
	\hline
\end{tabular}
\caption{Психологическая струкутра личности}
\end{table}

\textbf{Темперамент}~--- это характеристика человека, связанная с особеностью нервных процессов. Темперамент обуславливает интенсивность и скорость реагирования, степень эмоциональной возбудимости и особенности адаптациии к окружающей среде.

Различают 4 вида темперамента:
\begin{enumerate}
	\item \textbf{Холерик}~--- сильная, подвижная нервная система, но неуравновешеная (возбуждение преобладает над тормажением).
	\item \textbf{Сангвиник}~--- сильная, уравновешеная, подвижная нервная система.
	\item \textbf{Флегматик}~--- сильная, уравновешеная, но неподвижная нервная система.
	\item \textbf{Меланхолик}~--- слабая, неуравновешеная и неподвижная нервная система.
\end{enumerate}

\textbf{Характер}~--- это сочетание наиболее устойчивых, существенных особенностей личности, проявляющихся в поведении человека и его отношении к окружающим людям, себе и делу. На формирование характера существенно влияет система базовых ориентаций личности, которая складывается в первые годы жизни:
\begin{itemize}
	\item Ориентация на себя.
	\item Ориентация на других (на группу).
	\item Ориентайция на предметные аспекты активности.
\end{itemize}

\subsubsection{Акцентуация характера}
Это преувеличеное развитие отдельных черт характера. Были выделены следующие типы акцентуации характера:
\begin{itemize}
	\item Гипертивный. Всегда повышеное настроение.
	\item Дистимичный. Противоположный предыдущему. Полный пессимист.
	\item Циклотимный. Цикличность смены этапов подъема и спада.
	\item Эмотивный. Очень эмоциональный и чувствительный.
	\item Педантичный. Точный, аккуратный и т.д.
	\item Тревожный. С повышенным беспокойством насчет себя, близких и т.д.
	\item Застревающий. Долгое время фокусируется на проблеме.
	\item Демонстративный. Стремление выделится, не быть как все.
	\item Экзотированный. Похоже на предыдущее, но больше идет не от характера, а от природы. Часто деятели исскуства. Любят поговорить, влюбчивые, творческие и т.д.
	\item Возбудимый. Вспыльчивый, заносчивый.
\end{itemize}

\subsubsection{Способности}
Психологические особенности человека, обуславливающие успешное выполнение одной или нескольких видов деятельности. Предпосылкой развития способностей являются задатки человека~--- это анатомо-физиологические особенности строения мозга, нервной системы и системы анализаторов. Так же на развитие способностей существенно влияют социальные факторы, в том числе и интерес самого человека.

Различают общие способности и специальные. Общие характерны для всех людей и используются в различных видах деятельности. Специальные способности свойтвенны не всем и характеризуют успех в какой-то сфере.

\subsection{Типологии личности}
\subsubsection{Сенсорная типология}
Выделяют 3 типа людей в зависимости от ведущей системы восприятия (репрезентативная система):
\begin{itemize}
	\item визуальный (хорошая зрительная память, характерная сильная жестикуляция, в момент вспоминания информации взгляд обращен вверх),
	\item аудиальный (хорошая слуховая память, проговаривание слов при написании, смотрит в сторону при вспоминании информации),
	\item кинестетический (хорошая двигательная память, запоминает ощущения, чувтсва, движения, смотрит вниз при вспоминании информации).
\end{itemize}
\subsubsection{Типология доминирующего инстинкта}
\begin{table}
\begin{tabular}{|c|c|}
	Инстинкт & Тип личности\\
	\hline
	Самосохранения & Эгофильный\\
	\hline
	Продолжения рода & Генофильный\\
	\hline
	Исследования & Исследовательский\\
	\hline
	Свободы & Либертофильный\\
	\hline
	Доминирования & Доминантный\\
	\hline
	Сохранения достоинства & Дигнитофильный\\
	\hline
	Альтруизма & Альтруистический\\
	\hline
\end{tabular}
\caption{Типология доминирующего инстинкта}
\end{table}

\subsubsection{Конституциональная типология}
\textbf{Типология Э.Кречмера}\\

Выделяют различные типы телосложения и в соответствие им ставят типы личности.
\begin{itemize}
	\item Астеническое телосложение. Худощавые, высокие люди. \textbf{Шизотеники}. Замкнутые, интроверты, интелектуалы. Сложны в общении.
	\item Пикническое телосложение. Невысокий, полный. \textbf{Циклотеники}. Противоположны. Просты в общении и тд.
	\item Атлетическое телосложение. \textbf{Иксотемики}. Уверенный в себе, склонен к лидерству, стандартное, банальное мышление.
\end{itemize}

В совместимости людей можно выделить ряд уровней:
\begin{enumerate}
	\item ценностно-ориентационное единство, совместимость жизненных целей, ценностей, идеалов и т.д;
	\item согласованность функционально-ролевых ожиданий, те в представлении людей о своих ролях и функциях в конкретном союзе;
	\item совместимость индивидуальных психологических характеристик:
		\begin{enumerate}
			\item психо-физическая совместимость темпераментов,
			\item совместимость приобретенных черт характера
		\end{enumerate}
\end{enumerate}

\subsection{Формирование личности}
\subsubsection{Психо-социальная концепция развития личности Эриксона}
Центральными понятиями данной концепции явялются 
\begin{itemize}
	\item \textbf{групповая идентичность},
	\item \textbf{эго-идентичность}.
\end{itemize}

\textbf{Групповая идентичность} формируется с первых дней жизни человека, посколько ребенок ориентирован на включение в определенную соц. группу и начинает понимать мир, как эта группа. Но постепенно у ребенка формируется и \textbf{эго-идентичность}, те чувство устойчивости своего "<я">.

Формирование эго-идентичности~--- это длительный процесс, включающий ряд стадий. На каждой стадии человек сталкивается с кризисным противоречием, которое должен разрешить, что и определяет направление развития на данном возрастном этапе.

Эриксон выделял 8 стадий:
\begin{enumerate}
	\item Младенчество (до 1-го года). Базисное доверие к миру и людям.
	\item Ранний возраст (от 1-го до 3-х лет). Автономия, самостоятельность, опрятность.
	\item Дошкольный возраст (от 3-х до 6-ти лет). Инициативность, целеустремленность, активность.
	\item Школьный возраст (от 6-ти до 11-ти лет). Уверенность в себе.
	\item Юношеский возраст (от 11-ти до 20-ти лет). Цельная форма эго-идентичности. Человек находит свое "<я">. Признание окружающими.
	\item Молодость (от 20-ти до 25-ти лет). Чувство близости, интимности, любовь.
	\item Зрелость (от 25-ти до 50-ти лет). Творчество, любимая работа, воспитание детей, удовлетворенность жизнью.
	\item Старость (после 50-ти лет). Принятие себя, жизни, мудрость.
\end{enumerate}

Многие проблемы, осложнения развития являются следствием неразрешенности кризисных противоречий предыдущих периодов развития.

Каждый человек в процессе своего развития включается в различные соц. группы, в которых занимает определенные статусы. Наиболее характерный для данной личности статус по которому его выделяют окружающие называется \textbf{главным статусом}.

Все статусы делятся на предписанные (заданые от рождения, например, пол и национальность) и достигаемые (получаемые своими силами).

\textbf{Социальная роль}~--- это поведение, ожидаемое от того, кто имеет определенный социальный статус.

В процессе исполнения личность соц. ролей может возникнуть ролевое напряжение и конфликт (внутриличностный). Основными причинами могут быть:
\begin{enumerate}
	\item неадекватная ролевая подготовка,
	\item противоречия между требованиями различных ролей,
	\item несоответствие исполняемой роли интересам личности.
\end{enumerate}

\subsubsection{Действия с помощью которых может быть снижена ролевая напряженность}
\begin{itemize}
	\item Рационализация ролей. Безсознательный поиск неприятный сторон желаемой, но недостижимой роли.
	\item Разделение ролей. Временное изьятие из жизни одной из ролей.
	\item Регулирование ролей. Перенос ответственности за исполнение роли на других или на группу.
\end{itemize}

\section{ТУТ ПРОПУСК, ВСТАВЬ!!!!}

\subsection{Механизмы и типы межличностного восприятия}
В процессе межличностного восприятия существует ряд механизмов:
\begin{itemize}
	\item идентификация~--- отождествление себя с другим человеком,
	\item эмпатия~--- сочувствие, переживание другмоу человеку,
	\item рефлексия~--- осознание того, как собеседник воспринимает нас.
\end{itemize}

\subsubsection{Виды восприятия}
\begin{enumerate}
	\item Аналитическое. Каждый информативный элемент внешности человека связывается с наличием определенной психологической черты.
	\item Эмоциональное. Человеку приписываются те или иные черты на основе эмоционального отношения к нему.
	\item Социально-асоциативное. Человеку приписываются определенные черты на основе отнесения его к конкретной социальной группе.
	\item Перцептивно-асоциативный. Человеку приписываются те или иные черты по аналогии со вспоминаемым образом.
\end{enumerate}

\subsubsection{Эффекты восприятия}
\begin{enumerate}
	\item Эффект первичности и новизны. Эффект первичности срабатывает при общении с незнакомым человеком. Первая информация полученная о личности оказывает существенное влияние на формирование дальнейшего образа человека.
	
	Эффект новизны срабатывает при общении с уже знакомыми людьми. Последняя информация, полученная о человеке фиксируется в памяти, осмысливается и может привести к изменению мнения о нем.
	\item Эффект проецирования. Это перенос своих личностных особенностей на собеседника.
	\item Эффект стереотипизации. Стереотипы в общении чаще всего касаются групповой принадлежности индивида. С одной стороны они могут упрощать процесс восприятия другого человека, с другой приводить к формированию предубеждения.
	\item Эффект ореола. Ореол~--- это вспоминаемы образ, который мешает объективно оценить другого человека.
	\item Эффект отношения. При позитивном отношении к нам, мы склонны наделять человека позитивными качествами. При негативном~--- отрицательными.
	\item Эффект превосходства. Если собеседник превосходит нас по какому-либо критерию, то мы склонны его переоценивать. Если уступает, то мы его недооцениваем.
\end{enumerate}

Для того, чтобы правильно воспринять и оценить другого человека необходимо пронаблюдать за его поведением в следующих ситуациях:
\begin{itemize}
	\item Когда человек достигает значимой для него цели.
	\item Когда преодолевает трудности на пути к достижению желаемой цели.
\end{itemize}

Эти ситуации должны охватывать 3 основные сферы: общение, учение и труд.

\section{Общение, как взаимодействие}
\subsection{Трансактный анализ общения}
Автор Э. Берн. В данной концепции предполагается регулирование действий участников общения через согласование их позиций, учета характера ситуации и стиля взаимодействия.
\paragraph{Позиция}
Каждый участник общения может занимать одну из 3х позиций, которые условно обозначаются, как "<родитель">, "<взрослый"> и "<ребенок">. \textbf{Родитель}~--- это поведение в стандартных ситуациях, когда существует определенный алгоритм действия. Могут быть 2 разновидности: критикующий и опекающий родитель. Критикует постоянно указывает, наказывает, критикует и т.д. Опекающий родитель заботится ит.д.

\textbf{Взрослый}~--- это поведение в нестандартных ситуациях, когда нужно принимать решения, оценивать ситуацию, действовать сознательно.

\textbf{Ребенок}~--- отвечает за творчество, разрядку напряжения, получение приятных впечатлений. Также включается в действие, когда человек не чувствует уверенности или сил для решения проблемы. Выделяют 2 разновидности: бунтующий и послушный ребенок.

Берн полагал, что общение максимально эффективно, когда общение идет с одинаковых позиций (транзакции). Если транзакции не совпадают (пересекаются), то общение неэффективно.

\paragraph{Ситуация}
Также играет роль ситуация в общении. В разлицных ситуациях человек представляет себя с разных позиций и в разном стиле.
\paragraph{Стиль}
Различают 3 основных стиля взаимодействия:
\begin{itemize}
	\item Ритуальный.
	\item Манипулятивный.
	\item Гуманистический.
\end{itemize}

\subsection{Классификация взаимодействий}
В процессе жизнедеятельности люди вступают в разнообразные по характеру взаимодействия, наиболее распространенным является дихотомическое деление всех возможных видов взаимодействия на кооперацию и конкуренцию.

\section{Психологические аспекты конфликтных взаимодействий}
\textbf{Конфликт}~--- это резкое обострение противоречий и столкновение 2х или более участников в процессе решения проблемы, имеющей деловую или личную значимость для каждой из сторон.
$$
	\text{Конфликт}=\text{Участники}+\text{Объект}+\text{Конфликтная ситуация}+\text{Инцидент}
$$

\subsection{Классификация конфликтов}
\begin{itemize}
	\item По числу участников.
	\begin{itemize}
		\item Внутриличностный.
		\item Межличностный.
		\item Между личностью и группой.
		\item Межгрупповой.
	\end{itemize}
	\item По характеру протекания.
	\begin{itemize}
		\item Рациональные.
		\item Эмоциональные.
	\end{itemize}
	\item По проблемно-деятельностному признаку.
	\begin{itemize}
		\item Педогагический.
		\item Творческий.
		\item Политический.
		\item и т.д.
	\end{itemize}
\end{itemize}

Причины конфликтов бывают различны. Чаще всего межличностные конфликты возникают при несовпадении взгялдов, мировозрений и т.д.

\subsubsection{Виды межличностных конфликтов}
\begin{itemize}
	\item Конфликт безысходности (оба относятся отрицательно).
	\item Конфликт несовместимости (один относится положительно, а другой отрицательно, например, неразделенная любовь).
	\item Конфликт неопределенности (один не определился в своем отношении к другому).
	\item Конфликт влечения-боязни.
\end{itemize}

\subsubsection{Этапы протекания конфликтов}
\begin{enumerate}
	\item Предконфликтная стадия. Опоненты подгатавливают план действий, собирают информацию, оценивают ресурсы, осуществляется консолидация сил, участвующих в конфликте.
	\item Непосредственно конфликт. Характерно наличие инцидента, действия участников конфликта могут носить открытый или скрытый характер.
	\item Разрешение конфликта. Возможен при изменении конфликтной ситуации. Изменить ситуацию можно устранив причину конфликта. Также изменению конфликтной ситуации может способствовать вмешательство третей силы, создающее перевес одной из сторон, исчерпание ресурсов, устранение соперника.
\end{enumerate}

\subsubsection{Последствия конфликта}
\begin{tabular}{c|c}[h!]
Конструктивные & Деструктивные\\
\hline
Содействует внутригрупповой сплоченности, в случае межгрупповых конфликтов & Разрушение социальных групп\\
Привлечение внимания к интересам отдельной личности & Отвлечение внимания членов группы от решения насущных проблем\\
Формирование новых социальных связей & Разрушение социальных связей\\
\ldots & Способствует внутриличностной напряженности\\
& \ldots \\
\hline
\end{tabular}

Целесообразно разрешение конфликта на начальных этапах его возникновения. Необходимо соблюдение правил формального и неформального общения, важную роль играет личный пример руководителя и благоприятный психологический климат в коллективе.

\section{Психология делового общения}
\subsection{Организация публичного выступления}
Название выступления должно отражать его суть и носить рекламный характер. Обязательно нужно подготовить тезисы или конспект выступления. Необходимо изучить состав аудитории, осведомленность слушателей в обсуждаемом вопросе, интерес к теме выступления.

Не рекомендуется начинать выступление сразу, а сделать небольшую паузу, для установления визуального контакта с аудиторией. Первая фраза должна содержать приветствие. Первая часть выступления называется \textbf{зачином}, он должен быть занимательным и создавать эмоциональный контакт с аудиторией. Например, можно начать речь с красочного описания темы, яркого эпизода, привести афоризм или цитату. Важнейшим условием поддержания внимания к выступлению является новая, неизвестная слушателям информация или оригинальная интерпретация известных фактов.

Для поддержания внимания аудитории можно использовать психологические механизмы: сопереживание, доверительность, убежденность и эмоциональность оратора. Рекомендуются свободные и естественные жесты. Жестикуляция должна осуществляться выше пояса и обеими руками. Не должно быть рук за спиной или в карманах. Темп речи должен быть умеренным, речь должна содержать паузы для осмысления сказанного и формулирования вопросов. Переодически необходимо устанавливать визуальный контакт с аудиторией.

Для привлечения внимания также можно использовать юмор, подходящий под ситуацию и тему.

Заключение выступления должно быть связано с главной идеей, быть оптимистичным по духу. Хорошо воспринимаются выступления, заключение которых перекликается с началом.

Часто после выступления задаются вопросы. Существует 5 основных групп вопросов:
\begin{enumerate}
	\item Закрытые вопросы. Подразумевают ответ "<да"> или "<нет">. Используются, чтобы получить согласие или подтверждение того, что сказано ранее.
	\item Открытые вопросы. Например, как/почему/зачем. Задаются, когда нужно получить дополнительные сведения, либо выяснить мотивы, позицию выступающего.
	\item Риторические вопросы. Не дается прямого ответа, их цель вызвать новые вопросы, указать на нерешенные проблемы.
	\item Переломные вопросы, задаются когда слушатели получили достаточно информации по одной проблеме и хотят переключится на другую.
	\item Вопросы для обдумывания, вынуждают выступающего размышлять и комментировать то, что было сказано.
\end{enumerate}

\subsection{Психологические аспекты ведения деловой беседы}
Процесс ведения деловой беседы включает ряд этапов:
\begin{itemize}
	\item Подготовка к деловой беседе, установление места и времени встречи. Предпологается составление плана беседы, выделение основных задач, сбор необходимой информации о собеседнике, отбор аргументов для защиты своей позиции, выбор подходящей стратегии общения (открытая или закрытая).
	\item Начало беседы. Включает встречу и вступление в контакт. Недопустимы следующие варианты начала беседы: неуверенность, обилие извинений, неуважение, пренебрежение, агрессивность по отношению к партнеру. Для начала беседы чаще всего пользуются 4мя основными приемами:
		\begin{itemize}
			\item метод снятия напряжения (несколько приятных фраз личного характера, может легкая шутка),
			\item метод зацепки (необычный вопрос, сравнение, личные впечатления),
			\item метод стимулирования воображения (постановка ряда вопросов, которые должны быть рассмотренны в процессе беседы),
			\item метод прямого подхода (непосредственный переход к делу без вступления, подходит для кратковременных и не очень важных деловых контактов).
		\end{itemize}
	\item Постановка проблемы. Первый вопрос должен быть интересным, но недискуссионным. Лучше начинать с наиболее выполнимого предложения. Необходимо привести способ передачи информации в соответствии с мотивами и уровнем информированности собеседника. Не употреблять слов с двойным значением и фраз, которые можно неверно истолковать. Важно стремиться перейти от монолога к диалогу, комбинировать различные виды вопросов. Необходимо наблюдать за невербальными реакциями собеседника и соответственно гибко менять свое поведение.
	\item Фаза нейтрализации замечаний собеседника. Необходимо выслушать сразу несколько возражений не перебивая собеседника. Не спешить с ответом, не поняв суть возражения. Доказательство бессмысленности замечаний или эмоциональная реакция на них как правило приводят к росту конфронтации. Для нейтрализации возражений используют следующие приемы:
		\begin{itemize}
			\item ссылка на авторитет,
			\item цитата,
			\item сравнение,
			\item переформулировка.
		\end{itemize}
	\item Фаза поиска и принятия приемлимого решения. Может осуществляться в стиле сотрудничества, либо в форме авторитарного принятия решения одним из партнеров и добровольным или вынужденным подчинением другого партнера. Фиксация договоренности и завершение беседы. Итоги должны быть резюмированны, полезно составить оффициальный протокол решения.
\end{itemize}

\subsection{Деловые дискуссии при принятии решений}
\subsubsection{Деловое совещание}
Участвуют 7-9 человек. Тема обсуждения должна быть заранее определена, чтобы участники могли подготовится и продумать свои предложения. Целесообразно пространственное расположение участников в форме круглого стола. Деловое совещание предпологает возможность критики предметных позиций участников, а не их личных особенностей.
\subsubsection{Мозговой штурм}
Был предложен А. Осборном. Группа разбивается на 2 подгруппы: генератор идей и критики. Первую часть дискуссии генератор порождает как можно большее количество идей (даже невозможные, неудачные, необоснованные) по поводу поставленной проблемы. Вторую часть вступают в действие критики, которые анализируют идеи, обсуждают спорные и неудачные варианты, ищут пути улучшения, принимают лучшие.
\subsubsection{Метод синектики}
Синектика~--- соединение разнородного. Разработал данный метод У. Гордон. В группе выделяются синекторы~---- это люди, наиболее активно заявляющие свои позиции (оптимально 5-7 человек), в их задачу входит выдвижение противоположных/взаимоисключающих предложений по поводу поставленной проблемы; далее в дискуссию включаются другие члены группы, которые должны видеть возникшие крайности с тем, чтобы всесторонне их проанализировать. В итоге принимается решение удовлетворяющее всех. Предусмотренное обязательное использование приемов, основанных на аналогии. Существует 4 вида аналогии:
\begin{itemize}
	\item прямая (поиск подобных проблем),
	\item личная (поставить себя на место проблемного объекта),
	\item символическая (попытка переформулирования задачи),
	\item фантастическая (предположение, что мы имеем абсолютно любые средства для решения проблемы).
\end{itemize}

\section{Социально-психологические процессы  малых группах}
\subsection{Понятие и классификация малых групп}
Малая группа~--- это немногочисленная по составу общность людей, связанных совместной деятельностью и тесными межличностными отношениями. Отличительными особенностями малых групп являются:
\begin{enumerate}
	\item Пространственное и временное соприсокосновение членов группы.
	\item Наличие организационного начала в группе (формальный или неформальный лидер).
	\item Наличие эмоциональных отношений в группе.
	\item Наличие общей цели в совместной деятельности.
	\item Выработка специфической групповой культуры.
\end{enumerate}
\subsubsection{Классификация малых групп}
Классификация формальных-неформальных групп, предложенная Э. Мейо.
\begin{itemize}
	\item Формальные группы. Имеет устойчивый состав и четкую иерархию ролей. Например, класс, группа, рабочий коллектив, семья.
	\item Неформальные группы. Нет устойчивого состава, свободное членство, нет зафиксированной иерархии.
\end{itemize}

Классификация, предложенная Хайменом.
\begin{itemize}
	\item Группы членства. Группы, в которые мы включаемся.
	\item Референтные группы. Эталонные группы, которые являются для личности образцом.
\end{itemize}

Классификация, предложенная Коли.
\begin{itemize}
	\item Первичная группа. Каждый общается с каждым.
	\item Вторичная группа. Не все члены группы общаются или даже знакомы друг с другом.
\end{itemize}

\subsection{Групповое давление и групповая сплоченность}
Конформность~--- это характеристика позиции личности, относительно позиции группы, связана с принятием или непринятием группового мнения. Степень выраженности комфорности зависит от ряда факторов:
\begin{enumerate}
	\item характеристики самой личности,
	\item взаимотношения личности и группы,
	\item сущность решаемой проблемы.
\end{enumerate}
Различают внешнюю и внутренюю конформность. \textbf{Внешняя}~--- демонстративное согласие с мнением группы, которое внутрене человек не разделяет. \textbf{Внутреняя}~--- принятие мнения группы и следование ему, даже при отсутствии группы. Также различают нормативное и информационное влияние группы. \textbf{Нормативное}~--- влияние большинства, групповое мнение является нормой, обязательной для выполнения. \textbf{Информационное}~--- влияние меньшенства и мнение группы является информацией для размышления.

\subsubsection{Групповая сплоченность}
Исследуется процесс превращения номинальной группы в психологическую общность людей.

\textbf{Социометрический подход} предложил Морена. Метод основан на анкетировании. Все вопросы в анкете только в открытой форме, сплоченность связывается с таким уровнем развития взаимоотношений в группе, когда в ней высок процент выборов, основанных на взаимной симпатии.

\textbf{Стратометрический подход} предложил Петровский. Выделил три слоя отношений:
\begin{enumerate}
	\item Внешний. Формирование отношений в группе.
	\item Средний. Ценностно-ориентационное единство.
	\item Внутрений. Общие цели.
\end{enumerate}

\subsection{Лидерство и стиль руководства}
\subsubsection{Теории происхождения лидерства}
Существует ряд теорий объясняющих происхождения лидерства:
\begin{itemize}
	\item Теория черт. Лидер~--- человек, обладающий определенным набором качеств.
	\item Ситуативная теория. В конкретной ситуации человек проявляется необходимое качество, которым другие не обладают, в результате чего человек становится лидером.
	\item Системная теория. Лидер выдвигается группой, как субъект управления групповыми процессами.
\end{itemize}

\subsubsection{Стили руководства}
\begin{itemize}
	\item Авторитарный стиль. Жесткое единоличное принятие руководителем всех решений, жесткий контроль за выполнением решений с угрозой наказания, отсутствие внимания к личности сотрудников.
	\item Демократический стиль. Управленческие решения принимаются на основе обсуждения проблем с учетом мнений и инициатив сотрудников. Контроль за выполнением принятых решений осуществляется и руководителем и самими сотрудниками. Присутствует внимание к интересам личности.
	\item Либеральный стиль. Все могут высказывать свои позиции, но реального их учета и согласования нет. Отсутствует контроль за выполнением принятых решений.
\end{itemize}
Можно выделить ряд рекомендаций для руководителя по формированию развивающего климата в коллективе:
\begin{enumerate}
	\item Открытость и непринужденность общения.
	\item Понимание, способность войти в положение другого человека.
	\item Акцент на положительных качествах личности.
	\item Использование разнообразных поощрений. Желательно публично.
	\item Использование предложений, а не приказов.
\end{enumerate}

\section{Психологические аспекты формирования имиджа}
Имидж~--- в переводе с английского обозначает образ, ореол. Это сложившийся в массовом сознании и имеющий характер стереотипа эмоционально окрашенный образ кого-либо или чего-либо.
\subsection{Имидж делового человека}
Имидж личности включает не только естественные свойства, но и специально созданные. Он говорит как о внешнем облике человека, так и о его внутренем мире. Можно выделить 3 группы качеств, которые способствуют созданию позитивного имиджа:
\begin{enumerate}
	\item Привлекательность, Эмпатичность, Рефлексивность.
	\item Характеристики, формируемые в процессе воспитания, в тч нравственные ценности.
	\item Качества, связанные с жизненным и профессиональным опытом личности, что позволяет обострить интуицию.
\end{enumerate}

Во многих случаях имидж~--- это результат умелой ориентации в конкретной ситуации и заодно правильного выбора модели поведения. Каждый человек осваивает те модели, которые приносят ему успех. Для выражения своей позиции одновременно используются разнообразные средства:
\begin{itemize}
	\item вербальные,
	\item невербальные.
\end{itemize}

Этот комплекс признаков образует паттерн поведения, который воспринимается, как единое целое. Целостность нарушается, когда входящие в ее состав элементы противоречат друг другу.

Различают этикетные и стратегические модели поведения. Этикетные модели свойственны каждой профессии. Стратегические модели позволяют достигать конкретных целей. 

При выборе модели поведения учитывают объективные и субъективные факторы. К объективным можно отнести соответствие закону, институциональным нормам и морали. Соответствие поведения поло-ролевым стереотипам.

Субъективные факторы: цель, которую ставит личность, самокритичная оценка собственных возможностей использования конкретной модели поведения.

Для имиджа делового человека важна тактика общения, большое значение имеет вариативность и маневренность поведения. Так же можно использовать механизмы психологического воздействия:
\begin{itemize}
	\item Привязанность,
	\item Симпатия,
	\item Доверие,
	\item Уважение,
	\item Манеры.
\end{itemize}

\textbf{Групповой имидж}~--- обобщенный образ соц. группы, разделяемый представителями других соц. групп. В качестве содержательных элементов групповой имидж включает такие параметры, как национальный менталитет, темперамент итд.

\subsection{Формирование корпоративного имиджа}
Это символический образ организации, создаваемый процессе субъект-субъектного взаимодействия. Еще в начале 20го века было высказано мнение о том, что обществу свойственно институционализировать личность и персонифицировать институты. Решение о создании или смене имиджа организации всегда принимается высшим руководством. В отличие от индивидуального имиджа корпоративный всегда направлен на внешнее восприятие. Корпоративный имидж всегда является функциональным. При его помощи решаются конкретные прогматические задачи:позиционирование организации и побуждение к действию.

Различают благоприятный и нейтральный имидж фирмы.

Основой корпоративного имиджа явялется организационная культура: совокупность используемых неформальных процедур или преобладающая в организации философия, обуславливающая предпочтения, относительно способов достижения организационных целей. Элементами организационной культуры являются:
\begin{enumerate}
	\item Стиль отношений между сотрудниками, в том числе между руководителем и подчиненными.
	\item Стиль принятия решений.
	\item Стиль управления проблемами и конфликтами.
	\item Стиль осуществления изменений.
	\item Стиль отношений с внешними компонентами микро и макро среды
	\item Ценности организации.
	\item Наиболее распространенные отношения сотрудников к целям организации.
\end{enumerate}

Формирование организационной культуры~--- это обязательный процесс, который может протекать стихийно или целенаправленно.

\section{Введение в педагогику}
\subsection{Становление педогагогической науки}
Педагогика (греч. pais~--- ребенок, ago~--- вести). Первые попытки осмысления практики воспитания с учетом потребностей общества относятся к эпохе античности. Высказывания о целях, задачах, содержании и средствах обучения и воспитания занимали важное место в работах Демокрита, Платона и Аристотеля. Древние греки установили ряд принципов, которые необходимо учитывать в процессе обучения:
\begin{itemize}
	\item психологические факторы,
	\item возрастные особенности,
	\item единство умственного, нравственного и физического воспитания,
	\item и т.д.
\end{itemize}
Конченой целью хорошего воспитания считалось нравственное достоинство и добрадетель.

Первым педогагическим трудом считается работа М. Квинтилиана "<Об образовании оратора">.

Педогагические идеи в эпоху средневековья развивались в религиозном русле. Существовало 3 типа школ:
\begin{itemize}
	\item приходские (церковные),
	\item монастырские,
	\item кафедральные.
\end{itemize}
Основными предметами были философия и богословие.

Для эпохи Возрождения характерно освобождение от религиозных догматов, возраждение интереса к личности и ее повседневной жизнедеятельности. Развивается система светского образования на основе активного научного развития.

Мыслители эпохи Просвещения развивали идеи о решающей роли образования личности.

Основоположником педагогики считается Я. Каменский. 

XIX в. считается эпохой реформаторской педагогики. Представители различных течений выступали за свободное развитие индивидуальности человека, разработку новых организационных форм и методов обучения, изменение содержания образования. И. Гербарт выделил в педагогике 2 важные части: дедактика и теория воспитания. Говорил о решающей роли интереса в обучении, о воспитывающем характере обучения и т.д. А. Дистервег ввел принцип культуросообразности образования, те учета в процессе образования особенностей культуры, истории, экономики и т.д. характерных для страны и народа.

В России основоположником педагогики считается Ушинский. Он создал психолого-педогагическую концепцию в которой обосновал детерминированность воспитания социально-экономическими условиями жизни людей. Различал преднамеренные и непреднамеренные факторы, влияющие на личность, особое значение придавал традициям, культуре, нравственному воспитанию. В 20-30е гг. в России были разработаны лучшие учебники по педагогике. В 1919г. под руководством С. Шацкого была открыта первая опытная станция по народному образованию. Проводились исследования роли микросреды формирования личности. Популярным было тестирование в школах. К концу 30х гг. число конкретных исследований заметно уменьшилось.

Начиная с 60х. гг. в Советской педагогике активно исследуется опрос о взаимосвязи деятельности и формирования личности (деятельностный подход). 

В современной педагогике значительное внимание уделяется разработке проблем дидактики, применению новых информационных технологий в образовании, разрабатываются новые методы и средства обучения.

\subsection{Объект и предмет педагогики}
Педагогика~--- это наука о специально организованной, целенаправленной и систиматической деятельности по формированию личности.
\subsection{Основные категории педагогики}
\begin{tabular}{|c|c|c|c|c|c|c|c|d}
\multicolumn{3}{|c|}{Воспитание} & \multicolumn{3}{|c|}{Образование}  & \multicolumn{2}{|c|}{Обучение}  \\ 
\hline 
Мировозрение & Отношение & Поведение & Знания & Умения & Навыки & Преподавание & Учение \\ 
\hline 
\end{tabular} 
\subsection{Структура педагогики}
\begin{enumerate}
	\item Общая педагогика. Исследует общие закономерности обучения и воспитания, разрабатывает методы педогагического исследования, историю науки и т.д.
	\item Возрастная педагогика. Исследует особенности обучения и воспитания на различных возрастных этапах.
	\item Дидактика. Теория образования и обучения.
	\item Теория и методика воспитания. Занимается вопросами приобщения человека к культуре, формирование его ценностных ориентаций, норм поведения.
	\item Дефектология. Изучает особенности развития аномальных людей.
\end{enumerate}

\section{Психолого-дидактическая сущность процесса обучения}
\subsection{Общее представление о дидактике, дидактические принципы}
Дидактика(греч.~--- поучающий, относящийся к обучению). Понятие введено в 1613 г. в Германии, где дидактику рассматривали, как искусство обучения. В современной дидактике выделяют 2 взаимосвязанных аспекта.
\begin{itemize}
	\item Теоретический. Обосновывает, что из достижений современной культуры должно включаться в содержание образования и каким требованиям должна отвечать образованная личность. Исследуется влияние возрастных и психологических факторов на процесс обучения.
	\item Нормативно-прикладной. Связан с разработкой конкретных программ, организационных форм и методов обучения.
\end{itemize}

\subsubsection{Дидактические принципы}
\begin{enumerate}
	\item Научность обучения. Содержание образования должно составлять систему знаний, установленных наукой, знакомить с методами и историей науки.
	\item Доступность обучения. Материал должен быть доступен для сознательного усвоения учащимся, должен активизировать умственную деятельность.
	\item Систематичность и последовательность обучения. Преподование учебных дисциплин должно осуществляться в строгом логическом порядке.
	\item Связь теории с практикой обучения. Обеспечивает более глубокое усвоение научных знаний и формирует умение их применять.
	\item Сознательность и активность учащихся в обучении. Знания являются результатом собственной позновательной деятельности личности, осуществляемой под руководством педагога.
	\item Наглядность обучения. Создает чувственную основу для владения абстрактными понятиями.
	\item Прочность усвоения знаний. Обеспечивает длительное сохранение в памяти системы основных понятий, их постепенное углубление посредством самостоятельного применения на практике.
	\item Творческий характер обучения. Может реализовываться с помощью подбора определенных методов обучения и создания благоприятного психологического климата в процессе обучения.
	\item Учет сенситивных периодов развития. Те периодов наиболее чувствительных к усвоению определенных знаний.
	\item Учет социо-культурного контекста развития. Необходимо учитывать особенности экономики, культуры, традиции, обычаи конкретного общества.
\end{enumerate}

\subsection{Методы обучения}
Метод(греч.~--- путь к чему-либо). Под методом обучения понимают взаимодействие педагога и учащихся, направленное на достижение определенной образовательной цели. В зависимости от характера позновательной деятельности учащихся выделяют 2 группы методов обучения:
\begin{itemize}
	\item \textbf{методы традиционного обучения}~--- передача готовых знаний(объяснительно-иллюстративный, репродуктивный методы); недостатком этих методов можно назвать усредненный объем и темп подачи материала, плохой контроль за получаемыми знаниями; \textbf{программируемое обучение}~--- усовершенствование методов традиционного обучения, зародилось в 60е гг. XX в. на стыке педагогики, психологии и кибернетики;
	\item \textbf{методы проблемного обучения}~--- их суть в активном поиске и усвоении учащимися новых знаний(проблемное изложение материала, частично-поисковый метод или курсовой проект, исследовательский метод, деловые игры); проблемное обучение ставит своей задачей развитие творческого мышления и способностей личности, а также формирование умений в ходе активного поиска и самостоятельного решения проблем; основными условиями успешного проблемного обучения считаются:
	\begin{enumerate}
		\item необходимо вызвать интерес у учащегося,
		\item необходимо определить посильность работы с возникающими проблемами,
		\item необходимость создать комфортный стиль общения между преподавателем и учащимися, когда возможна свобода выражения своих мыслей и взглядов.
	\end{enumerate}
\end{itemize}
В дидактике используются различные методы исследования:
\begin{itemize}
	\item наблюдение учебного процесса,
	\item беседы с учащимися и преподавателями,
	\item анкетирование и эксперименты,
	\item психологическое тестирование,
	\item анализ работ учащихся,
	\item математические методы.
\end{itemize}
Применение этих методов позволяет выделить эмпирические закономерности обучения. Для расскрытия внутренних механизмов действия этих закономерностей применяются методы компьютерного моделирования, функционального, структурного и генетического анализа сложных систем.

\section{Восспитание в педагогическом процессе}
\subsection{Сущность, принципы и методы воспитания}
\textbf{Воспитание}~--- целенаправленный процесс, связанный с формированием мировозрения человека, его отношения к объектам окружающей действительности и поведения.

\subsubsection{Принципы воспитания}
\begin{enumerate}
	\item Индивидуальный дифференцированный подход к воспитанию.
	\item Сочетание требовательности и контроля с уважением достоинства личности.
	\item Акцент на положительных качествах личности.
	\item Воспитание в группе через коллектив.
	\item Воспитание в процессе совместной деятельности.
\end{enumerate}

\subsubsection{Методы воспитания}
\begin{enumerate}
	\item Убеждение, беседа. В качестве убеждения могут выступать как речевые, так и невербальные приемы(тембр голоса, жестикуляция). Наличие группы, отсутствие, авторитет убеждающего и т.д.
	\item Пример. Непосредственный(лично знакомые люди) и косвенный(знаменитые люди, ученые, абстрактные знакомые). Пример должен быть значим и понятен для человека, относится к сфере его интересов.
	\item Упражнения. Упражнения должны быть интересны, систематичны. Важно заинтересовать человек, чтобы упражнения выполнялись добровольно.
	\item Поощрение. Бывает формальное и неформальное поощрение.
	\item наказание. Формальное и неформальное. Рекомендуется применять метод строго на индивидуальном уровне, нельзя наказывать публично. Крайнее воздействие, не рекомендуются строгие и жестокие наказания.
\end{enumerate}

\subsection{Проблемы семейного воспитания}
Семья выполняет ряд важных функций в социализации человека:
\begin{itemize}
	\item \textbf{мировозренческая}~--- формирование ценностей, идеалов, целей человека и т.д.,
	\item \textbf{транслирующая}~--- передача и интерпретация для личности социальных норм общества и социальных групп,
	\item \textbf{регулятивная}~--- на основе принятых в семье норм обуславливает определенный тип поведения,
	\item \textbf{статусная}~--- человек получает некоторые статусы, близкие к статусам членов его семьи.
\end{itemize}
Семья способствует развитию индивидуальности, формируя следующие социально-психологические качества:
\begin{itemize}
	\item Умение взаимодействовать с разными людьми в различных жизненных ситуациях.
	\item Способность к самоутверждению и самореализации.
	\item Способность сочувствовать, сопереживать, содействовать другим.
	\item Способность любить близких людей и доверять им.
	\item Уверенность в собственной нужности.
\end{itemize}
\subsubsection{Стили семейного воспитания}
\begin{enumerate}
	\item \textbf{Авторитарный}. Жесткое воспитание с применением наказаний.
	\item \textbf{Демократичекий}. Сочетание требовательности и контроля с эмоциональным принятием ребенка.
	\item \textbf{Отстраненный}. Своеобразный отказ родителей от воспитания под предлогом того, что этим должны заниматься педагоги-профессионалы.
	\item \textbf{Детоцентриский}. Ребенок в центре внимания, выполняются все его желания и просьбы.
	\item \textbf{Прагматический}. Ориентация для получения выгоды для себя.
\end{enumerate}

Родительская позиция, оказывающая оптимальное воздействие на развитие личностного потенциала ребенка, обладает следующими свойствами:
\begin{itemize}
	\item Адекватность. Наиболее близкое к объективному отношение к ребенку.
	\item Динамичность. Это способность изменять методы и формы, воздействуя на ребенка, применительно к его возрастным особенностям, конкретным ситуациям и условиям жизни семьи.
	\item Прочность. Направленность воспитательных усилий в будущее, в соответствии с теми требованиями, которые ставит перед ребенком дальнейшая жизнь.
\end{itemize}

\subsection{Конфликты поколений}
Поколение~--- интервал времени 20-25 лет.

Проблемы взаимоотношения поколений можно рассматривать в 2х аспектах: 
\begin{itemize}
	\item Вертикальный аспект. В его основе лежит последовательность смены поколений в истории определенного периода, что позволяет представить общую картину передачи культурного наследия, трансляцию социо-культурного опыта.
	\item Горизонтальный аспект. Все живущие поколения действуют одновременно и между ними складываются различные взаимоотношения.
\end{itemize}