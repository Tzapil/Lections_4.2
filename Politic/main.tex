\section{Введение}
\subsection{Преподаватель}
\textbf{Тарасова Галина Николаевна}

\section{Сущность политики}
\subsection{Происхождение политики в обществе}
\textbf{Политика}~--- это вид целенаправленной деятельности. Политика, как особая сфера общественной жизни появляется на таком этапе развития общества, на котором утрачивается естественное согласие и появляется потребность в согласованном поведении. \textbf{Естественное согласие}~--- это такое согласие соц. группы, при котором индивид не отделяет своих интересов от интересов группы и следовательно его не следует принуждать к соблюдению соц. норм.

\subsection{Подходы к изучению политики в обществе}
\begin{itemize}
	\item Социологический. Политика изучается, как социальный феномен и во взаимосвязи с другими соц. явлениями. Например в Марксизме было дано определение "<Политика~---- это концентрированное выражение экономики">. Это говорит о том, что основой политики всегда является экономика и политика вторична по отношению к ней.
	\item Этический (нормативно-ценностный). Это подход в рамках которого рассматривается нравственная основа политики. В рамках этого подхода изучается соотношение политики и морали. Формируется нравственный идеал политики.
	\item Содержательный (субстанциональный).
	\item Институциональный. Акцент ставится на изучение гос. институтов. Все элементы общества рассматриваются, как части единого целого.
\end{itemize}
Можно дать несколько определений политике: Политика~--- это способ организации общественной жизни с помощью власти. Это процесс достижения и поддержания целостности системы, в условиях постоянного столкновения интересов соц. субъектов. Это управление обществом элитой этого общества.

\subsection{Функции политики в обществе}
\begin{itemize}
	\item Поддержание целостности соц. системы.
	\item Формирование целей для всего общества.
	\item Организация масс и мобилизация ресурсов для достижения целей.
	\item Обеспечение порядка в обществе (обычно с помощью правоохранительных органов).
	\item Распределительные (обеспечивается перераспределение некоторых соц. благ для некоторых соц. категорий).
	\item Политическая социализация.
\end{itemize}

\section{Политология, как наука о политике}
Становление относится к периоду к.XIX-н.XX вв. Для появления этой науки были сформированы следующие условия:
\begin{itemize}
	\item появление политических систем нового типа в Западной Европе и Америке (демократические системы),
	\item успехи в социально-гуманитарном знании,
	\item до 20-го века специалисты не могли придти к единому мнению о том, чем эта наука должна заниматься (существовали даже различные понятия политики и различные направления, которые претендовали на звание отдельных наук).
\end{itemize}

С середины XX в. происходит институциализация политологии, как самостоятельной науки. В 1948 г. по инициативе ЮНЕСКО был проведен международный коллоквиум "<По проблемам политической науки">. На этом коллоквиуме было решено использовать словосочетание "<политическая наука"> в единственном числе, основная предметная область которой:
\begin{itemize}
	\item проблемы теории политики,
	\item исория политических учений,
	\item функционирование политических институтов,
	\item действие субъектов политической власти,
	\item прикладные исследования,
	\item проблемы муждународных отношений.
\end{itemize}

В 1949 г. была создана "<Международная Ассоциация Политической Науки">, которая обратилась ко всем странам ООН с просьбой способствовать развитию и распространению политической науки.

В 1955 г. была создана "<Ассоциация Политической Науки"> в СССР.

Функции политологии:
\begin{itemize}
	\item позновательная,
	\item функция рационализации политических решений,
	\item функция прогнозирования,
	\item функция политической социализации,
	\item мировозренческая.
\end{itemize}

\section{Власть в обществе}
\subsection{Социальная сущность власти}
\textbf{Власть}~--- некая абстрактная сущность, обладание которой дает право на повелевание. Этот подход соответствует идеалистическому подходу к власти.

\textbf{Власть}~--- универсальная форма социальных отношений, которая возникает в результате взаимодействия субъектов и выражается в том, что субъект или группа имеют право и возможность принимать решения, которые становятся обязательными для других индивидов, группы или общества в целом.

Властные отношения всегда демонстрируют ассиметрию социального влияния.

Власть существует в 2х основных формах:
\begin{itemize}
	\item Естественная форма власти, возникает в любой более или менее устойчивой соц. группе на основе естественного неравенства между людьми. Эта власть неформализованна и имеет персонифицированный характер.
	\item Социальный институт, в процессе развития соц. отношений власть становится соц. институтом. На практике это означает, что в обществе появляются специализированные учереждения или организации, в которых концентрируются управленческие статусы, приобретение которых дает право на повелевание. Такая власть утрачивает персонифицированный характер.
\end{itemize}

\textbf{Политическая власть}~--- право, способность и возможность реализовать политическую волю через политические институты.

Политическая власть обладает свойством \textbf{суверенитета} или \textbf{суверенности}. Те преобладанием над всеми другими формами власти.

Политическая власть состоит из:
\begin{itemize}
	\item Субъектов власти (лидеры, элита, государство, партии),
	\item Объектов власти (граждане, группы людей, классы, общество),
	\item Источников власти (то, что власть использует для сохранения, например, сила, богатство, знания и т.д.),
	\item Ресурсы (то, что власть использует для выполнения своих функций, например, материальный, демографические и т.д. Самыми важными считаются \textbf{страх, убеждение и интерес}),
	\item Функции (функции политики, те регулирование, управление, координация и т.д.).
\end{itemize}


\subsection{Легитимность политической власти}
\textbf{Легитимность} (франц. Законность)~--- предпологает, что объект власти признает за субъектом власти право на властвование. Теоретический анализ этого понятия связан с именем М. Веббера. Он говорил, что легитимность опирается на определенные социо-культурные источники. Он выделил 3 трипа легитимности.
\begin{itemize}
	\item Традиционная легитимности. Основанием считал способность людей принимать привычное, понятное и способность это воспроизводить. Яркий образец~--- передающаяся по наследству власть.
	\item Харизматическая легитимность. Основанием является способность человека к эмоциональному восприятию окружающей реальности. Те общество готово принимать власть, субъектом которой является харизматичный лидер. Сам Веббер считал, что это лучший тип легитимности. Он считал, что такие лидеры наиболее сильные и способны радикально менять общество.
	
	В современном мире существует феномен "<исскуственная харизма">.
	\item Рационально-легальный. Основа~--- способность человека к рациональной оценке окружающей действительности. Общество готово принимать власть, сформированную на основе законных демократических процедур.
\end{itemize}

\section{Лидерство}
Лидерство~--- это разновидность приоритетного постоянного и легитимного влияния со стороны людей, занимающих более высокую властную позицию, которое распространяется на общество или группу.

Политический лидер~--- человек, выступающий с политической инициативой, независимо от занимаемой должности. Принимает ответственность за свои решения.

\subsection{Теории лидерства}
\begin{itemize}
\item Теория черт.

Лидерами становятся люди с выдающимися положительными качествами.
\item Ситуационная теория.

Лидерами становятся люди попавшие в благоприятную ситуацию.

\item Теория окружения или последователей.

Большую роль в развитии лидера играет окружение.
\end{itemize}

\subsection{Функции лидеров}
\begin{itemize}
	\item Инструментальное лидерство.
		\begin{itemize}
			\item Технологическая функция. Лидер должен быть представителем той сферы деятельности, которой он руководит.
			\item Кодификаторская функция. Лидер~--- образец выполнения законов, правил, норм, этикета итд.
			\item Представительская функция.
			\item Административная функция. 
		\end{itemize}
	\item Эмоциональное лидерство.
		\begin{itemize}
			\item Психотерапевтическая функция или "<функция демонстрации оптимизма">.
			\item Формирование дуалистической структуры на принципе этноцентризма.
		\end{itemize}
\end{itemize}

\section{Стили лидерства}
Факторы влияющие на выбор стиля:
\begin{itemize}
	\item Внешние.
	\begin{enumerate}
		\item Исторический.
		\item Организационно-управленческий. В пространственно растянутых управленческих структурах существует зависимость: если субъекты власти отдаляются от объектов власти, то возникает вероятность более жесткой системы управления. Если создаются механизмы сближения субъектов и объектов, то распространяется демократический стиль.
		\item Экономический. Если в экономике проявляется тенденции к накоплению, то более вероятен становится жесткий стиль управления.
	\end{enumerate}
	\item Внутрение.
	\begin{enumerate}
		\item Личность руководителя. Тип управления зависит от общей грамотности и характера руководителя.
		\item Уровень компетенции ближайшего окружения руководителя.
		\item Соотношение между гуманитарной и технической подготовкой у руководителей.
	\end{enumerate}
\end{itemize}

Политическая элита~--- это отдальная соц. группа и она имеет все черты соц. группы. Внутри элиты существует внутреняя система взаимодействия, внутреняя стратификация, существует внутренее самосознание (самоидентификации). Внутри группы существует идеология правящего меньшинства. Для этой идеологии характерен миссионерский дух (выполнение особой миссии). Существуют объективные и субъективные причины формирования этой идеологии. Среди объективных причин можно выделить особый род деятельности группы.

\subsection{Теории элит}
Разработкой подобных теорий занимались итальянские ( Моска и Парет ) и австрийские ( Михельс ) в конце XIX в начале XX вв. Парето и Моска считали, что в элиту попадают люди одаренные от рождения и заслуживающие право находиться среди элиты. Среди факторов они выделяли: богатство, воинская доблесть, церковный сан.

Парето указывал на возможность существования нескольких элитарных обществ, в том числе и контрэлиты.

Самой яркой считается теория Михельса. Она называлась "<Теория железного закона олигархии">. Михельс не верил в устойчивое и долгосрочное существование демократии. Он утверждал, что в любой демократической структуре появляется группа людей~--- элита, которая обладает большим авторитетом и влияет на все решения группы. В конечном счете эта группа становится правящей и общество перерождается в олигархию.

\subsubsection{Ценностная теория элит}
\begin{itemize}
	\item Элита~--- самый ценный элемент общества, поскольку состоит из лучших представителей этого общества.
	\item Существование элиты не противоречит интересам общества. 
	\item Появление элиты не результат естественного отбора, а результат сознательной селекции в обществе.
\end{itemize}
\subsubsection{Плюралистические теории элит}
\begin{itemize}
	\item Отрицание существования единой управленческой элиты. В каждой сфере общественной жизни формируется своя элита, действующая только в рамках своей сферы.
	\item Элиты находятся под контролем групп на которые опираются("<Материнских групп">). Этот контроль сдерживает отрыв элиты от общества и действие закона олигархии.
	\item Группы интересов/группы давления.
\end{itemize}

\subsection{Способы рекрутирования в элиту}
Выделяют 2 основных способа:
\begin{itemize}
	\item Закрытый(способ гильдий). Основные черты:
		\begin{itemize}
			\item Ограниченая социальная база для набора.
			\item Отсутствие открытой политической конкуренции.
			\item Медленная профиссиональная карьера.
			\item Часто существует негласный патриархальный принцип отбора.
		\end{itemize}
		Сильные стороны:
		\begin{itemize}
			\item Стабильность
			\item Устойчивость
			\item Высокий уровень бюракратического проффесионализма членов
		\end{itemize}
		Слабые стороны:
		\begin{itemize}
			\item Тенденция к застою
			\item Тенденция к старению элиты
		\end{itemize}
	\item Открытый(антрепренерская). Основные черты:
		\begin{itemize}
			\item Широкая база для набора.
			\item Открытая политическая конкуренция, которая дает возможность продвижения молодых, амбициозных людей и идей.
			\item Непредсказуемость элиты. Возможность попадания в элиту непрофессиональных людей.
		\end{itemize}
	\item Номенклатурный способ. Таким способом формировалась элита СССР. Это частный случай закрытого способа.
\end{itemize}
Номенклатурная система советского общества стала разрушаться с 80-х гг. В 1989г. были проведены первые альтернативные выбборы депутатов на всесоюзный съезд депутатов КПСС. На съезд было выбрано огромное количество людей не закрепленных в номенклатурной элите страны того времени. Они стояли у истоков разрушения старой системы и создания новой. Позже многие закрепились в новой элите страны. В конце 80-х по инициативе Горбачева была задумана программа НТТМ (научно-техническое творчество молодежи). Молодежным структурам была разрешена свободная предпринимательская деятельность. Именно отсюда стала формироваться новая экономическая элита. В 1990г. отменили 6 статью конституцию СССР:"<КПСС~--- ядро политической системы советского общества">. За отмену этой статьи боролись очень долго.

Современная элита имеет ненаменклатурный характер, хотя представители бывшей советской элиты долго вермя занимали/занимают руководящие должности.

\section{Политическая жизнь общества}
Системный подход к изучению политики сформировался в середине XX вв. Согласно этому подходу все политические элементы общества рассматриваются во взаимосвязи друг с другом, как части единого целого.

\textbf{Политическая система}~--- совокупность учереждений и организаций, опирающихся на соответствующую им политическую структуру в совокупности представляющих собой политическую самоорганизацию общества.
\subsection{Политическая система общества}
\subsection{Государство как центральный институт политической системы}
\subsection{Основные разновидности политических режимов}
\subsection{Политические партии и группы интересов}