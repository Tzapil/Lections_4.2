\section{Введение}
\subsection{Преподаватель}
\textbf{Степанов Е. Г.}

\subsection{Список литературы}
Любые учебники потолще.
\begin{enumerate}
	\item Белов. "<Безопасность жизнидеятельности">.
	\item Сакова. "<Правовы и организационные основы БЖД">.
\end{enumerate}

\section{Основы БЖД}
БЖД является компиляцией многих областей науки, но основой является предмет "<Охрана труда и техника безопасности">. Также туда входят:
\begin{itemize}
	\item основы физиологии,
	\item экология,
	\item оценка рисков.
\end{itemize}

\subsection{Основные понятия}
БЖД изучает опасности и способы защиты человека в любых условиях. \textbf{Опасность} является основным понятием БЖД. \textbf{Опасность}~--- это любое явление, угрожающее жизни и здоровью человека.

\textbf{Предмет изучения} БЖД:
\begin{itemize}
	\item современное состояние и факторы среды обитания,
	\item принципы обеспечения безопасного взаимодействия человека со средой обитания,
	\item последствия воздействия на человека травмирующих, вредных и поражающих факторов,
	\item средства и методы обеспечения безопасности техники и технологических процессов,
	\item мероприятия по защите человека в условиях ЧС и ликвидация последствий аварий, стихийных бедствий итд,
	\item правовые и организационные основы БЖД.
\end{itemize}

\textbf{Практические задачи} БЖД~--- выбор путей и методов защиты человека. Разработка средств защиты и снижение влияния на окружающую среду.

\textbf{Научные задачи} БЖД~--- это теоритический анализ и разработка методов идентификации опасных и вредных факторов, генерируемых элементами среды обитания.

\textbf{Объект изучений} БЖД~--- это комплекс явлений и процессов в системе "человек-среда обитания" негативно воздействующих на человека и природную среду. Система "человек-среда обитания" многовариантна (например, "человек-бытовая среда, человек-рабочая среда, человек-машина").

\textbf{Основной постулат} БЖД~--- это "<Аксеома потенциальной опасности">. Это потенциальное свойство процесса взаимодействия человека со средой обитания. Все действия человека, а особенно связаные с техническими средствами, кроме полезных свойств таят в себе опасные и вредные факторы.

\textbf{Вредный производственный фактор}~--- это факторы среды и трудового процесса, воздействие которого на работающего при определенных условиях (интенсивность, длительность и тд) может вызвать профессиональные заболевания, временное или стойкое снижение работоспособности, повысить частоту заболеваний, а также привести к нарушению здоровья потомства. По природе различают:
\begin{itemize}
	\item физические ([температура, влажность, скорость воздуха]микроклимат, [электромагнитные и др поля]наличие различных полей, [ультрозвуковое, лазерное, инфрокраснове]различные излучения, [ифразвук, шум]шум, различная пыль, освещенность, механические воздействия),
	\item химические (различные вредные вещества, а также вещества биологической природы, получаемые химическим или биохимическим синтезом),
	\item биологические (вирусы),
	\item трудового процесса
		\begin{itemize}
			\item тяжесть труда (характеристика, учитывающая воздействие на двигательный аппарат, органы дыхания и кровообращения),
			\item напряженность труда (характеристика трудового процесса, учитывающая нагрузку на центральную нервную систему, органы чувств и эмоциональную сферу работника).
		\end{itemize}
\end{itemize}

\textbf{Опасный производственный фактор}~--- это фактор среды или трудового процесса, который может быть причиной острого заболевания или внезапного ухудшения состояния здоровья или смерти. Граница между вредными и опасными факторами очень размыта и любой вредный фактор в перспективе может стать опасным.

Опасные и вредные факторы бывают:
\begin{itemize}
\item природные,
\item антропогенные.
\end{itemize}

\textbf{Риск}~--- частота возникания опасности. Чтобы потенциальная опасность стала реальной необходимо соблюдейние некоторых условий, которые называют причинами.

В нашей стране от различных опасностей неестественной смертью ежегодно умирают больше полумиллиона человек, основными причинами риска являются:
\begin{itemize}
\item производственный травматизм.
\end{itemize}
