\section{Введение}
\subsection{Преподаватель}
\textbf{Степанов Евгений Генадьевич}

\subsection{Список литературы}
Любые учебники потолще.
\begin{enumerate}
	\item Белов. "<Безопасность жизнидеятельности">.
	\item Сакова. "<Правовы и организационные основы БЖД">.
\end{enumerate}

\section{Основы БЖД}
БЖД является компиляцией многих областей науки, но основой является предмет "<Охрана труда и техника безопасности">. Также туда входят:
\begin{itemize}
	\item основы физиологии,
	\item экология,
	\item оценка рисков.
\end{itemize}

\subsection{Основные понятия}
БЖД изучает опасности и способы защиты человека в любых условиях. \textbf{Опасность} является основным понятием БЖД. \textbf{Опасность}~--- это любое явление, угрожающее жизни и здоровью человека.

\textbf{Предмет изучения} БЖД:
\begin{itemize}
	\item современное состояние и факторы среды обитания,
	\item принципы обеспечения безопасного взаимодействия человека со средой обитания,
	\item последствия воздействия на человека травмирующих, вредных и поражающих факторов,
	\item средства и методы обеспечения безопасности техники и технологических процессов,
	\item мероприятия по защите человека в условиях ЧС и ликвидация последствий аварий, стихийных бедствий итд,
	\item правовые и организационные основы БЖД.
\end{itemize}

\textbf{Практические задачи} БЖД~--- выбор путей и методов защиты человека. Разработка средств защиты и снижение влияния на окружающую среду.

\textbf{Научные задачи} БЖД~--- это теоритический анализ и разработка методов идентификации опасных и вредных факторов, генерируемых элементами среды обитания.

\textbf{Объект изучений} БЖД~--- это комплекс явлений и процессов в системе "человек-среда обитания" негативно воздействующих на человека и природную среду. Система "человек-среда обитания" многовариантна (например, "человек-бытовая среда, человек-рабочая среда, человек-машина").

\textbf{Основной постулат} БЖД~--- это "<Аксеома потенциальной опасности">. Это потенциальное свойство процесса взаимодействия человека со средой обитания. Все действия человека, а особенно связаные с техническими средствами, кроме полезных свойств таят в себе опасные и вредные факторы.

\textbf{Вредный производственный фактор}~--- это факторы среды и трудового процесса, воздействие которого на работающего при определенных условиях (интенсивность, длительность и тд) может вызвать профессиональные заболевания, временное или стойкое снижение работоспособности, повысить частоту заболеваний, а также привести к нарушению здоровья потомства. По природе различают:
\begin{itemize}
	\item физические ([температура, влажность, скорость воздуха]микроклимат, [электромагнитные и др поля]наличие различных полей, [ультрозвуковое, лазерное, инфрокраснове]различные излучения, [ифразвук, шум]шум, различная пыль, освещенность, механические воздействия),
	\item химические (различные вредные вещества, а также вещества биологической природы, получаемые химическим или биохимическим синтезом),
	\item биологические (вирусы),
	\item трудового процесса
		\begin{itemize}
			\item тяжесть труда (характеристика, учитывающая воздействие на двигательный аппарат, органы дыхания и кровообращения),
			\item напряженность труда (характеристика трудового процесса, учитывающая нагрузку на центральную нервную систему, органы чувств и эмоциональную сферу работника).
		\end{itemize}
\end{itemize}

\textbf{Опасный производственный фактор}~--- это фактор среды или трудового процесса, который может быть причиной острого заболевания или внезапного ухудшения состояния здоровья или смерти. Граница между вредными и опасными факторами очень размыта и любой вредный фактор в перспективе может стать опасным.

Опасные и вредные факторы бывают:
\begin{itemize}
	\item природные,
	\item антропогенные.
\end{itemize}

\textbf{Риск}~--- частота возникания опасности. Чтобы потенциальная опасность стала реальной необходимо соблюдейние некоторых условий, которые называют причинами.

В нашей стране от различных опасностей неестественной смертью ежегодно умирают больше полумиллиона человек, основными причинами риска являются:
\begin{itemize}
	\item производственный травматизм (от 14 до 16 тыс. ч.),
	\item ДТП (39-40 тыс. ч.),
	\item наркомания (80 тыс. ч.),
	\item самоубийства (55-56 тыс. ч.),
	\item алкогольные отравления (33-35 тыс. ч.),
	\item туберкулез (29-30 тыс. ч.),
	\item утопления (15-16 тыс. ч.),
	\item врачебные ошибки (98-100 тыс. ч.).
\end{itemize}
$$
	R =\frac{n}{N}
$$
где $n$~--- число событий с летальным исходом по какой-либо причине, а $N$~--- число людей, котороые имеют или могут иметь место в данной ситуации.

Риски в зависимости от получившейстя величины делять на 3 группы:
\begin{itemize}
	\item $R<10^{-6}$. Риск допустимый.
	\item $R>10^{-3}$. Риск недопустимый (неприемлимый).
	\item $10^{-6}<R<10^{-3}$. Переходная область. Самая большая часть. Большинство рисков находится именно в этой области.
\end{itemize}

\section{Принципы, методы и средства обеспечения безоспастности}
\subsection{Принципы и методы обеспечения безопасности}
\textbf{Принцип}~--- идея, основная мысль.

Классификация по признаку реализации:
\begin{itemize}
	\item ориентирующие,
		\begin{itemize}
			\item снижения опасности,
			\item ликвидации опасности,
			\item деструкции,
			\item системности
			\item и тд.
		\end{itemize}
	\item организационные,
		\begin{itemize}
			\item защиты времени,
			\item информации,
			\item нормирования,
			\item несовместимости,
			\item отбора кадров
			\item и тд.
		\end{itemize}
	\item технические,
		\begin{itemize}
			\item блокировки,
			\item герметизации,
			\item защиты расстоянием,
			\item слабого звена,
			\item и тд.
		\end{itemize}
	\item управленческие,
		\begin{itemize}
			\item контроля,
			\item обратной связи,
			\item стимулирования,
			\item ответственности
			\item и тд.
		\end{itemize}
\end{itemize}

\textbf{Метод}~--- это способ достижения поставленной цели, исходящий из знания наиболее общих закономерностей.

Классификация:
\begin{itemize}
	\item разделение в пространстве и времени \textbf{ноксосферы}(пространство, где существует или переодически возникает опасность) и гомосферы (зон, где находится человек).
	\item нормализация ноксосферы путем исключения опасностей.
	\item совокупность приемов и средств, направленных на адаптацию человека к соответствующей среде и повышению его защищенности.
\end{itemize}

\subsection{Средства обеспечения безопасности}
\begin{itemize}
	\item Конструктивные.
	\item Организационные.
	\item Технические.
	\item и тд.
\end{itemize}

Средства обеспечения безопасности это конкретная реализация соответствующиех принципов и методов.

\section{Безопасность человека в производственной среде}
\textbf{Охрана труда}~--- система сохранения жизни и здоровья работников в процессе трудовой деятельности, которая включает:
\begin{itemize}
	\item правовые,
	\item соц-экономические,
	\item организационно-технические,
	\item санитарно-гигиенические,
	\item лечебно-профилактические,
	\item ребиалитационные
	\item и иные предприятия.
\end{itemize}

\textbf{Производственная санитария}~---  это совокупность мероприятий и средст, направленных на защиту работников.

\textbf{Техника безопасности}~--- комплекс защитных мероприятий и средст, исключающих воздействие на работника опасных производственных факторов.

\textbf{Профессиональные заболевания}~--- это заболевания, причиной которых стали вредные производственные факторы.

\subsection{Методы анализа производственного травматизма}
\begin{itemize}
	\item монографический,
	\item топографический,
	\item групповой,
	\item статистический.
\end{itemize}

\section{Основы физиологии и психологии труда}
\subsection{Физиология труда}
Изучает закономерности изменений функциональной системы организма в процессе труда для разработки рациональных режимов труда, рациональных рабочих поз и движений.

Основным понятием физиологии труда явялется работоспособность человека. Работоспособность человека~--- это способность функциональной системы организма выполнять требуемую работу за определенный период времени с заданным качеством. Работоспособность имеет 3 основные фазы:
\begin{itemize}
	\item врабатывание~--- приспособление к условиям и требованиям работы (15-20 минут),
	\item высокая устойчивая работоспособность~--- достигается наивысшая производительность труда при минимальных затратах организма (при работе с вредными и опасными условиями труда для поддержания высокой работоспособности вводятся дополнительные регламентированные перерывы не менее 5, но не более 20 минут; длится 2-4 часа),
	\item утомление~--- снижение возможностей и резервов организма, возникает чувство усталости, сонливости и т.д.(обычно утомление снимается перерывами, обеденными перерывами и т.д., если силы не восстанавливаются к следующей рабочей смене, то имеет место \textbf{переутомление}).
\end{itemize}
После обеденного перерыва работоспособность имеет 3 фазы, но реализуется на более низком уровне, а в ночное время работоспособность еще меньше, поэтому ночные смены короче.
\subsection{Классификация форм труда}
\begin{itemize}
	\item Формы труда со значительно двигательной активностью, высокими энергозатратами (от 4 тыс. до 6 тыс. ккал в сутки). Высокие нагрузки на опорно-двигательный аппарат, дыхательную и сердечно-сосудистую систему. Преимущество такой формы труда~--- это высокая двигательная активность, способствующая активизации обменных процессов в организме.
	\item Форма труда с незначительной двигательной активностью, невысокими энергозатратами (до 2 тыс. ккал в сутки). С нагрузками на внимание, память, мышление и эмоциональную сферу деятельности.
	\item Групповые и конвеерные технологии. Монотонный труд.
	\item Работа в автоматизированном и полуавтоматизированном производстве с работой в напряжениях ожидания и наблюдения.
	\item Механизированные формы труда. Нагрузка на отдельные группы мышц, с высокими требованиями к точности и координации движений, монотонность работы, снижение интелектуальной активности к концу рабочей смены.
\end{itemize}

\subsection{Психология труда}
Изучает процессы памяти, внимания, мышления, индивидуальные свойства личности, особенности темперамента. Психология труда изучает особые состояния, такие как монотония, психическое пресыщение и стресс.

\section{Вредные вещества}
\textbf{Вредное вещество (ВВ)}~--- вещество при контакте которого с организмом человека, в случае нарушения техники безопасности, может вызвать производственные травмы, профессиональные заболевания или отклонения в состоянии здоровья обнаруживаемые современными методами, как в процессе контакта с ними, так и в отдаленные сроки настоящего и будущих поколений.

Классификация:
\begin{itemize}
	\item Промышленные яды. Используются в производстве (например, краски, пары бензина итд).
	\item Ядохимикаты для сельского хозяйства.
	\item Лекарственные средства.
	\item Бытовые химикаты.
	\item Биологические растительные и животные яды.
	\item Боевые отравляющие вещества.
\end{itemize}

\textbf{ПДУ}~--- предельно допустимый уровень, это уровень фактора, который при воздействии на человека изолированно или при взаимодействии с другими факторами в течении рабочего времени ежедневно или в течении всего трудового стажа не вызывает у него и его потомства биологических изменений, в том числе скрытых и временно компенсированных, а также психологических нарушений (снижение умственных способностей итд).

Группы вредных веществ:
\begin{itemize}
	\item Общетоксические.
		\begin{itemize}
			\item Нервные.
			\item Кровяные.
			\item Печеночные.
		\end{itemize}
	\item Раздражающие.
	\item Прижигающие.
	\item Сенсибилизирующие (алергики).
	\item Канцерогенные (вызывают раковые заболевания).
	\item Мутагенные.
	\item Ферментные.
\end{itemize}

Виды отравлений:
\begin{itemize}
	\item Общие отравления. Яд всасывается в кровь и разносится по организму. Часто яды действуют на определенные органы.
	\item Местное действие. Воздействие только в районе попадания яда. Часто кожа, глаза, слизистые.
\end{itemize}

Отравленния могут протикать в острой и хронической формах. \textbf{Острое} часто бывает при авариях, когда в среду попадает огромное количество отравляющих веществ. Те острое отравление характеризуется кратковременным воздействием.

\textbf{Храническое} отравление происходит постепенно в результате поступления в организм небольшого количества вещества регулярно.

Различают действия вредных веществ на организм \textbf{изолированное} (очень редкое) и \textbf{комбинированное}. Комбинированное действие разделяют на \textbf{однонаправленное} и \textbf{разнонаправленные}.

\section{Звуковые и механические коллебания}
Виды:
\begin{itemize}
	\item Шум,
	\item Вибрация,
	\item Ультразвук,
	\item Инфразвук.
\end{itemize}
Звук в воздухе распространяется со скорость 344 м/с при нормальной температуре и давлении. Сам процесс распространения называют звуковой волной. Шум в отличии от звука вызывает неприятные ощущения и может принести вред здоровью. Человек слышит ограниченный звуковой диапозон от 16 Гц до 20 кГц (в реальности большинство людей слышат от 20-25 Гц и до 16 кГц). Инфоразвуком называют коллебания ниже 16 Гц. Стандартная частота для исследований звука $f_{\text{ст.}}=1000$ Гц. Это наиболее хорошо слышимая частота.

Основные физические параметры: звуковое давление ($P$, Па), интенсивность ($I$, Вт/$m^2$), частотная характеристика ($f$, Гц).

\textbf{Звуковое давление}~--- разница между мгновенным значением давления в возмущенной среде и средним в невозмущенной.
В санитарных нормах обычно устанавливаются 2 величины:
\begin{itemize}
	\item Порог слышимости:$P_0=2*10^{-5}$ Па.
	\item Болевой порог:$P_{\text{б.п.}}=2*10^2$ Па.
\end{itemize}

Минимальная интенсивность звука на стандартной частоте $I_0=10^{-12}$ Вт/$m^2$, а максимальная интенсивность(болевой порог) $I_{\text{б.п.}}=10^2$ Вт/$m^2$. Часто переходят к лагорифмической шкале и интенсивность звука считают в Б или дБ. За 0 была принята минимальная интенсивность звука. Возрастание интенсивности в 10 раз(на порядок) воспринимается, как увеличение громкости в двое.

Частотные параметры шума оценивают по спектру, который принято делить на октавы. \textbf{Октава}~--- это диапозон частот, когда верхняя граница вдвое превышает нижнюю. Впервые выделять октавы стали еще древние греки. Слух реагирует не на абсолютный, а на относительный прирост частот. Увеличение частоты вдвое воспринимается, как увеличение тона на октаву. Всего существует 9 октав.

Шум нормируют, тк доказано, что каждый лишний дБ снижает эффиктивность труда. Возникает куча заболеваний при долговременном и регулярном воздействии интенсивного шума. Шанс потери или ухудшения звука повышается на 2\% на каждый дБ. Шумы бывают широкополосные(спектр шире одной октавы) и тональные(уже одной октавы, одна частота или узкий диапозон преобладает над другими частотами). По временным характеристикам шумы бывают постоянными(малоизменяющимися) и непостоянные(более распространенные). Непостоянные в свою очередь подразделяются на коллеблющиеся во времени(сирены), прерывистые(ступенчатые) и импульсные. 

Для сравнения по частотам и интенсивности используется эквивалентный уровень звука.

Основные средства и методы борьбы с шумом:
\begin{itemize}
	\item борьба с первоисточником,
	\item закрыть источник шума звукоизолирующим чехлом (звукоизоляция),
	\item снижение шума на пути его распространения (звукопоглощение),
	\item средства индивидуальной защиты,
	\item ограничение по времени.
\end{itemize}

\subsection{Ультразвук}
Это коллебания с частотой выше порога слышимости. ПО спектральным характеристикам делится на низкочастотный(16-20кГц), среднечастотный(125-250кГц) и высокочастотный(1-$1{,}5$МГц). лучше всего в воздухе распространяется низкочастотный ультразвук. Распростроняется не только в воздухе. 

Высокочастотный ультразвук используется в основном для целей анализа и контроля, изучения мелких деталей на дефекты. В медицине для лечения, диагностики и хирургии. Источники бывают ручные и станционарные.

Основные способы защиты от ультразвука:
\begin{itemize}
	\item Дистанционное управление.
	\item Автоблокировка.
	\item Избежание.
	\item и т.д.
\end{itemize}

\subsection{Инфразвук}
Коллебания ниже порога слышимости. 

Воздействие на организм: инфразвук субъективно воспринимается, как физическая нагрузка (появляется эффект тяжести и вскоре усталость). Возникают вестибулярные расстройства, падает острота слуха.

Наименее благоприятный для человека диапозон 4-15Гц. Эти частоты воспринимаются как боль. Возможно возникновение эффекта резонанса с тканями, костьми и т.д. 

Инфразвук распространяется на огромные расстояния и очень сложен для экранирования. Наиболее эффективный метод: борьба с первоисточником.

\subsection{Вибрации}
Коллебательный процесс, вызываем переодическим смещением центра тяжести различных элементов машин, приспособлений и сооружений.

Классификация:
\begin{itemize}
	\item Общая вибрация, передающуюся через опорные поверхности на тело человека.
	\begin{itemize}
		\item Общие вибрации 1й категории. Транспортная вибрация. Источники: автомобили, тракторы и прочая техника.
		\item Общая вибрация 2й категории. Транспортно-технологическая вибрация. Источники6 спец. техника, например экскаваторы, краны, бетоноукладчики, катки и т.д.
		\item Общая вибрация 3й категории. Технологическая вибрация. Источники: стационарные машины, например станки, вентиляторы, насосы и т.д.
		\item Общая вибрация в жилых помещениях от внешних источников. Городской транспорт, близкие промышленные предприятия, спец. техника.
		\item Общия вибрация в жилых и общественных зданиях от внутрених источников. Лифты, вентиляторы, кондиционеры, насосы и другое оборудование.
	\end{itemize}
	\item Локальная вибрация, передается через руки. 
	\begin{itemize}
		\item От ручного механизированного инструмента, а также органов ручного управления различным оборудованием.
		\item От ручного немеханизированного инструмента.
	\end{itemize}
\end{itemize}
Если вибрация передается на ноги сидящего человека или предплечья, которые контактируют с рабочей поверхностью столов, то вибрация тоже считается локальной.

Воздействие на организм: головная боль, бессоница, раздражительность, утомление. 

Наиболее распространены в настоящее время заболевания, вызванные локальной вибрацией. В первую очередь нарушение сосудистой системы (повышение или снижение чувствительности конечностей, пальцев), нарушения работы суставов.

Основные параметры вибраций:
\begin{itemize}
	\item Амплитуда смещения.
	\item Виброскорость.
	\item Виброускорение.
	\item Период колебаний.
	\item Частота колебаний.
\end{itemize}

Основные способы борьбы с вибрацией:
\begin{itemize}
	\item Борьба с источником.
	\item Уменьшение вибрации на пути распростронения.
	\begin{itemize}
		\item Виброгашение. Увеличение масс агрегата.
		\item Вибропоглощение.
		\item Виброизоляция.
	\end{itemize}
\end{itemize}

Виды механической опасности:
\begin{itemize}
	\item Движущиеся машины и механизмы.
	\item Подвижные части оборудования.
	\item Движущиеся изделия и заготовки.
	\item Острые кромки, заусеницы.
	\item Разрушающиеся конструкции.
	\item Статические и динамические перегрузки человека.
	\item Расположение рабочего места на значительной высоте.
	\item Повышенная запыленность.
	\item Обрушающиеся горные породы.
\end{itemize}

Способы защиты:
\begin{itemize}
	\item Дистанционное управление.
	\item Применение спец. оборудования и различных тормозных устройств.
	\item Использование оградительных устройств.
	\item Знаки безопасности.
\end{itemize}

\section{Электромагнитные поля}
