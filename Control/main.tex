\section{Введение}
\subsection{Преподаватель}
\textbf{Павлов Роман Владимирович}

\subsection{Список литературы}
\begin{enumerate}
	\item Бесекерский В. А., Попов Е. П. "<Теория систем автоматического регулирования">. Москва, 1975 г.
	\item Павлов Р. В. "<Основы теории управления">.
\end{enumerate}

Посмотреть описания RC-цепочек, в пособии Вишнякова.\\
Ряд Тейлора.

\section{Общие положения}
Теория автоматического управления явялется технической наукой общего применения. Она дает теоритическую базу для исследования и разработки систем автоматического управления.

\textbf{Операция управления}~--- это действие, направленное на правильное и высококачественное функционирование технического объекта. Они обеспечивают в нужный момент времени начало, порядок следования и прекращение отдельных действий. Выделяют необходимые ресурсы, задают нужные параметры самому процессу управления.

\textbf{Процесс управления}~--- это совокупность операций управления.

\textbf{Объект управления}~--- это совокупность технических средств, выполняющих определенный процесс, подлежащий управлению.

Типовая функциональная схема управления имеет вид:
Картинка 1.

Сигналы:
\begin{itemize}
	\item $g(t)$~--- задающее воздействие (например, педаль газа),
	\item $f(t)$~--- возмущающее воздействие (вредное воздействие, тряска на кочках),
	\item $e(t)$~--- сигнал разсогласования или ошибки (разность двух предыдущих входных сигналов),
	\item $z(t)$~--- сигнал главной обратной связи,
	\item $y(t)$~--- выходной сигнал (регулируемая величина).
\end{itemize}

Блоки:
\begin{itemize}
	\item 1. Задающее устройство. Преобразует входной сигнал (например, к масштабам и природе сигнала обратной связи).
	\item 2, 5. Сравнивающие устройства.
	\item 3. Преобразующее устройство.
	\item 6. Усилительное устройство.
	\item 7. Исполнительное устройство.
	\item 9. Измерительный элемент (датчик положения, скорости и тд).
	\item 10. Элемент главной обратной связи (преобразование сигнала для сапостовления с входным).
	\item 11. Объект управления (сам самолет, антена и тд).
	\item 4, 8. Корректирующие устройства (улучшают качества системы), причем 4~--- последовательное корректирующее устрйство, а 8~--- местная обратная связь.
\end{itemize}

\subsection{Классификация САУ}
Критерии классификации САУ:
\begin{itemize}
	\item Системы делят на:
		\begin{itemize}
			\item автоматической стабилизации (поддержание какой то величины на постоянном уровне),
			\item слежения (выходной сигнал повторяет входной),
			\item программного управления (реализация какого-либо закона управления).
		\end{itemize}
	\item Системы делят на:
		\begin{itemize}
			\item замкнутые,
			\item разомкнутые.
		\end{itemize}
		Настоящие системы управления всегда замкнутые, но в задачах анализа также изучаются разомкнутые системы.
	\item Делят на:
		\begin{itemize}
			\item автоматические,
			\item автоматизированные (функции обратной связи выполняет человек).
		\end{itemize}
	\item Делят на:
		\begin{itemize}
			\item линейные (описывается линейным управлением),
			\item нелинейные (нелинейным уравнением).
		\end{itemize}
		Обычно уравнения дифферинциальные.
	\item По характеру сигнала делятся на:
		\begin{itemize}
			\item непрерывные,
			\item имульсные,
			\item цифровые.
		\end{itemize}
	\item Системы делят на:
		\begin{itemize} 
			\item обычные,
			\item адаптивные (параметры системы подстраиваются под изменения среды).
		\end{itemize}
\end{itemize}

\subsection{Задачи теории САУ}
\begin{enumerate}
	\item Формирование функциональных и структурных схем САУ.
	\item Построение статических и динамических характеристик отдельных звеньев системы и системы в целом.
	\item Определение ошибок управления и показателей точности.
	\item Исследование устойчивости работы САУ.
	\item Оценка качественных показателей процесса управления.
	\item Синтез корректирующих устройств и оптимизация параметров САУ.
\end{enumerate}

\section{Математический аппарат описания систем автоматического управления}
Исчерпывающим описанием САУ является дифференциальное уравнение ее работы. В общем случае это \textbf{уравнение динамики}. Любое диффуравнение пишется относительно входных и выходных величин, если величины меняются, то и получается уравнение динамики. Если приравнять к 0 входной сигнал, то получим \textbf{уравнение статики}, те описание системы в установившемся режиме.

Диффуры получаются на основе изучения физ. законов системы. В общем случае эти диффуры являются нелинейными. Для упрощения анализа выполняется линеаризация. Существует большое количество способов линеаризации, например разложение в ряд Тейлора. Линейное диффуравнение может быть представлено следующим образом:
	$$
	a_n\frac{d^nx_2}{dt^n}+a_{n-1}\frac{d^{n-1}x_2}{dt^{n-1}}+\ldots+a_0x_2(t)=b_m\frac{dx^m}{dt^m}+\ldots+b_0x_1(t)
	$$
где $x_1$~--- входныя величина, а $x_2$~--- выходная. Это стандартный вид диффура, но в теории автоматического управления существует другая форма, принятая за стандарт, например для уравнения 3й степени:
	$$
	(T^2_2p^2+T_1p+1)x_2(t)=k(T_3p+1)x_1(t)
	$$
где k~--- коэффициент усиления, T~--- постоянная времени, а $p=\frac{d}{dt}$.

Это уравнение называют \textbf{первой формой записи САУ}. Кроме нее существует еще 2 формы, где отличается смысл параметр p.

Описание путем использования диффуров используется редко в связи со сложностью анализа. Для этого в теории САУ вводится понятие \textbf{передаточная функция} или функция передачи. Функцию передачи можно получить путем использования преобразования Лапласа:
	$$
	F(p)=\int_0^{\infty}f(t)e^{-pt}dt, p=a+jb
	$$
где f(t)~--- оригинал, а F(p)~--- изображение. Другая форма записи $F(p)=L(f(t))$. Это \textbf{вторая форма записи}.

После преобразования диффура уравнение примет вид:
	$$
	(T^2_2p^2+T_1p+1)X_2(p)=k(T_3p+1)X_1(p)
	$$
из этого уравнения можно найти отношение:
	$$
	W(p)=\frac{X_2(p)}{X_1(p)}=\frac{k(T_3p+1)}{T_2^2p^2+T_1p+1}
	$$
где $p=j\omega$. Тогда $W(j\omega)$. Это \textbf{третья форма записи}.

Это и называется \textbf{передаточной функцией}, как правило это отношение двух полиномов для конкретной системы (обычно степень числителя меньше степени знаменателя). Эта функция служит основой для теории САУ.

\subsection{Характеристики САУ}
При анализе САУ применяются различные методы оценки реакции САУ на внешние воздействия (какой-то входный сигнал). Например, для сравнения двух систем берут один входной сигнал, подаем на вход обоих систем и сравниваем выходные сигналы. Для сравнения необходимо, чтобы входные воздействия были стандартные (эталонный). Таких сигналов в теории САУ 3.
\subsubsection{Мощный короткий входной импульс (дельта-импульс)}
	\begin{gather*}
		\delta(t)=\left\{\begin{matrix}0, &t\neq0\\{}\infty, &t=0\end{matrix}\right.\\
		\int^{\infty}_{-\infty}\delta(t)dt=1
	\end{gather*}
	$\omega(t)$~--- \textbf{весовая функция}. Картинка 2.
	
	После применения преобразования Лапласа $L(\delta(t))=1$, тогда $X_2(p)=W(p)X_1(p)$, те $L(\omega(t))=W(p)$. Картинка 3.
	
	$x_2(t)=\int^t_0x_1(\tau)\omega(1-\tau)d\tau$~--- это интеграл свертки или интеграл Дюамеля.
	
\subsubsection{Единичная ступенька}
	$$1(t)=\left\{\begin{matrix}0, &t<0\\1, &t>=0\end{matrix}\right.$$
	Тогда $h(t)$~--- переходная характеристика.
	
	$$L(1(t))=\frac{1}{p}$$. Найти математическую связь между переходной и весовой характеристикой.
	
\subsubsection{СИнусоидальный сигнал}
	Частотные характеристики~--- это выражения и графические зависимости, выражающие реакцию системы на воздействие вида $\sin\omega{}t$. При передаче такого сигнала на вход САУ, выходной сигнал будет $x_2=A\sin(\omega{}t+\phi)$. Где $\phi$~--- начальная фаза второго сигнала или \textbf{сдвиг фаз}.
	
	Существует зависимость параметров выходного сигнала от частоты входного. Это и называют \textbf{частотными характеристиками}, те $A(\omega)$~--- это коэффициент усиления системы и $\phi(\omega)$~--- сдвиг фаз. Обе характеристики находят из \textbf{частотной передаточной функции}:$W(j\omega)=u(\omega)+jv(\omega)$ в Декартовых координатах и $W(j\omega)=A(\omega)e^{j\phi(\omega)}$.
	\begin{gather*}
		A(\omega)=\sqrt{u^2(\omega)+v^2(\omega)} \\
		\phi(\omega)=\arctan\frac{v(\omega)}{u(\omega)}
	\end{gather*}
	Частотные характеристики удобно представлять в графическом виде на осях $u$,$jv$. Картинка 6.
	
	Эта кривая наз. \textbf{Амплитудно-Фазовая характеристика} или \textbf{годограф}, те показывает зависимость коэфициента усиления и сдвига фаз от частоты сигнала. Это характеристика системы или звена.
	
	Возможно построить отдельно \textbf{Амплитодно-частотную характеристику} и \textbf{Фазово-частотную характеристику}, но эти графики как правило не строят, а чаще строят логарифмические графики. Для его построяния берут десятичный логарифм:
	$$
		L(\omega)=20\lg{}A(\omega)
	$$
	это Логарифмическая Амплитудно-Частотная Характеристика(ЛАЧХ). Величина измеряется в дБ.
	
	Это строится графически на осях $\omega$,$L(\omega)$. Частота откладывается в лагорифмическом масштабе. Картинка 7.
	
	Если частота изменяется на декаду (порядок), то логарифмическая хорактеристика меняется на число кратное 20 дБ (в зависимости от степени частоты). Наклон в графике стандартный. График уходит в отрицательную область, значит сигнал ослабляется, те Амплитуда<1 и логарифм<0.
	
	С помощью логарифмических характеристик можно легко исследовать САУ на устойчивость и др. Кроме того имея ЛХ отдельных звеньев легко получить ЛХ всей системы.

Первый сигнал~--- единичнй удар, второй~--- постоянное нажатие, а третий~--- постоянное изменение (вправо, влево).

\subsection{Сотавление исходных диффуравнений и передаточных функций}
В общем случае для системы можно записать несколько видов диффуров:
\begin{enumerate}
	\item $D(p)x(t)=Q(p)g(t)+N(p)f(t)$, где $x(t)$--- ошибка, $f(t)$~--- задающее значение, $g(t)$~--- возмущающее значение.
	\begin{gather*}
		D(p)=a_0p^n+a_1p^{n-1}+\ldots+a_{n-1}p+a_n\\
		x(t)=g(t)-y(t)
	\end{gather*}
	где $y(t)$~--- выходной сигнал
	Полином $D(p)$ имеет специфическое название \textbf{Характеристический полином} и часто используется для анализа системы.
	\item $D(p)y(t)=R(p)g(t)-N(p)f(t)$, где $y(t)$~--- выходня величина, а $R(p)=D(p)-Q(p)$
\end{enumerate}

Передаточные функции. Картинка 8.

Блоки:
\begin{itemize}
	\item ПУ~--- преобразующее устройство,
	\item ИЭ~--- исполнительый элемент,
	\item РО~--- регулируемый объект.
\end{itemize}
Здесь также введен новый сигнал $u(t)$~--- регулирующий сигнал. Для рассмотреного случая можно написать следующие формулы.

\begin{gather*}
	u(t)=W_{per}(p)x(t)\\
	y(t)=W_0(p)u(t)+W_f(p)f(t)
\end{gather*}

Здесь присутствует 3 передаточные функции: $W_{per}$~--- описывает звенья САУ, $W_0$~--- передаточная функция по регулирующему воздействию (описывает РО) и $W_f$~--- передаточная функция по возмущающему воздействию.
\begin{gather*}
	y(t)=W(p)x(t)+W_f(p)f(t) \\
	W(p)=W_0(p)W_{per}(p)=\frac{R(p)}{Q(p)}=\frac{Y(p)}{X(p)}
\end{gather*}
Последняя функция называется \textbf{передаточная функция разомкнутой системы}.

Теперь рассмотрим замкнутую систему, те введем обратную связь, для этого добавляем уравнение $x(t)=g(t)-y(t)$~--- уравнение замыкания.
\begin{gather*}
	y(t)=\frac{W(p)}{1+W(p)}g(t)+\frac{W_f(p)}{1+W(p)}f(t) \\
	x(t)=\frac{1}{1+W(p)}g(t)-\frac{W_f(p)}{1+W(p)}f(t)
\end{gather*}
Глядя на эти выражения можно ввести еще несколько стандартных передаточных функций:
$$\text{Ф}(p)=\frac{W(p)}{1+W(p)}=\frac{R(p)}{R(p)+Q(p)}=\frac{Y(p)}{G(p)}$$
Это \textbf{Передаточная функция замкнутой системы}.
$$\text{Ф}_x=\frac{1}{1+W(p)}=1-\text{Ф}(p)=\frac{X(p)}{G(p)}$$
Это \textbf{Передаточная функция замкнутой системы по ошибке}.
Существует \textbf{Передаточная функция замкнутой системы по возмущающему воздействию}. САМОСТОЯТЕЛЬНО!

\section{Типовые звенья САУ}
Рассмотрим наиболее распространенные звенья:
\subsection{Усилительное (пропрорциональное или безинерционное)} 
	Картинка 9.
	
	Дифференциальное уравнение выглядит очень просто $x_2(t)=k_1*x_1(t)$. Передаточная функция равна $W(p)=k_1$. Весовая функция $\omega(t)=k_1\delta(t)$, а $h(t)=k_1{}1(t)$. Графики нарисовать самостоятельно.
	
	Пример, любой усилитель, механическая передача, потенциометр.
	
\subsection{Апереодическое (инерционное звено первого порядка)}
	Диффур выглядит так $T\frac{dx_2}{dt}+x_2(t)=k_1x_1(t)$. Передаточная функция выглядит следующим образом $W(p)=\frac{k_1}{Tp+1}$. Весовая функция $\omega(t)=\frac{k_1}{T}e^{-\frac{t}{T}}$. $h(t)=k_1(1-e^{-\frac{t}{T}})$. Графики.
	
	\begin{gather*}
		W(j\omega)=\frac{k}{Tj\omega+1}=u(\omega)+jv(\omega)=\frac{k}{Tj\omega+1}\times\frac{1-Tj\omega}{1-Tj\omega}=\frac{k}{1+T^2\omega^2}-j\frac{kT\omega}{1+T^2\omega^2}\\
		\phi(\omega)=\arctan\omega{}T\\
		A(\omega)=\ldots
	\end{gather*}
	Картинка 10.

	Лагорифмические характеристики:
	$$L(\omega)=20\lg(\frac{k}{\sqrt{1+T^2\omega^2}})=20(\lg(k)-\lg(\sqrt{1+T^2\omega^2}))$$
	Картинка 11.

	Частота $\omega_c=\frac{1}{T}$ называют \textbf{сопрягающей}.

\subsection{Апереодическое звено 2го порядка}
	\begin{gather*}
		(T_2^2p^2+T_1p+1)x_2(t)=kx_1(t)\\
		W(p)=\frac{k}{T^2_2p^2+T_1p+1}
	\end{gather*}
	Для разложения полинома 2го порядка на произведение полиномов первого порядка, необходимо чтобы $T_1\leq2T_2$, тогда:
	$$W(p)=\frac{k}{(T_3p+1)(T_4p+1)}$$
	Таким образом это объединение двух апереодических звеньев первого порядка.
	\begin{gather*}
		A(\omega)=\frac{k}{\sqrt{(T_3^2\omega^2)(T_4^2\omega^2+1)}}\\
		\phi(\omega)=-\arctan{}T_3\omega-\arctan{}T_4\omega
	\end{gather*}
	В данном случае сопрягающих частот будет 2.\\
	Картинка 12.
\subsection{Колебательные и консервативные звенья}
	Для обоих звеньев диференциальное уравнение имеет тот же самый вид.
	$$(T_2^2p^2+T_1p+1)x_2(t)=kx_1(t)$$
	Но соотношение постоянных времени будет обратным $T_1<2T_2$. В этом случае данную передаточную функцию записыают в другом виде.
	$$W(p)=\frac{k}{T^2p^2+2\xi{}Tp+1}$$
	Где $T=T_2$, а $\xi=\frac{T_1}{2T_2}$~--- коэфициент затухания (от 0 до 1).\\
	Переходная характеристика выглядит так: Картинка 13\\
	Величина колебаний зависит от коэффициента затухания, чем он меньше, тем больше колебания.\\
	Лагорифмические характеристики выглядят так: картинка 14.
	
	Консервативное звено.\\
	Если задать коэфициент затухания $=0$, то графики будут выглядеть так: картинка 15.
	
\subsection{Интегрирующее}
	\begin{gather*}
		x_2(t)=k\int{}x_1(t)dt \\
		W(p)=\frac{k}{p}
	\end{gather*}
	Картинка 16.
\subsection{Дифференцирующее звено}
	\begin{gather*}
		x_2(t)=k\frac{dx_1}{dt}\\
		W(p)=kp
	\end{gather*}
	Картинка 17.

\section{Преобразование структурной схемы САУ}
Имеется САУ заданной структуры, состоящяя из определенного количество звеньев. Может возникнуть 2 задачи:
\begin{enumerate}
	\item На основе передаточных функций звеньев получить передаточную функцию всей системы. При этом используется 3 основных вида соединений звеньев.
	\item Преобразования структурной схемы. Используется 2 приема:
\end{enumerate}

\subsection{Основные виды соединений звеньев}
\subsubsection{Последовательное соединений}
Картинка 18.
$$W_{ekv}(p)=W_1(p)*W_2(p)$$
\subsubsection{Паралелльное соединение}
Картинка 19.
$$W_{ekv}(p)=W_1(p)+W_2(p)$$
\subsubsection{Обратная связь}
Картинка 20.
\begin{gather*}
	e=x\pm{}z\\
	y=e*W_1\\
	z=y*W_2\\
	W_{ekv}(p)=\frac{y}{x}=\frac{W_1(p)}{1\mp{}W_1(p)W_2(p)}
\end{gather*}
\subsection{Преобразования структурной схемы}
\subsubsection{Перенос сумматора}
Картинка 21.
\subsubsection{Перенос узла}
Картинка 22.

\section{Устойчивость САУ}
Под устойчивостью понимается способность системы сохранять свое состояние или принимать новое после окончания действия внешних сил.

Состояние системы определяется по выходному сигналу. Если выходной сигнал стремится к постоянному значению, то система устойчива, иначе неустойчива.

Картинка 23.

Для того, чтобы определить устойчивость необходимо определить теоретический выходной сигнал, те решить диффур (в общем виде и в частном, переходная составляющая, вынужденное решение, \textbf{УЗНАТЬ!!}). Для решения диффура в общем случае необходимо отсутствие правой части.
$$
	y(t)=C_1e^{p_1t}+C_2e^{p_2t}+\ldots
$$
Где $C_1, C_2,\ldots$~--- коэфициенты, найденные из начальных условий, $p_1,p_2,\ldots$~--- конри характеристического уравнений $D(p)=0$. Вид выходного сигнала \textbf{зависит от вида корней}.

Рассмотрим частный случай решения характеристического уравнения:
\begin{enumerate}
	\item 
	Действительный корень.
	\begin{gather*}
		p_1 = +\alpha\\
		p_2 = -\alpha
	\end{gather*}
	График 24.
	
	В этом случае на устойчивый сигнал подходит второй корень.
	\item
	Мнимый корень.
	\begin{gather*}
		y(t) = C_1e^{\pm\alpha{}t}\sin(\beta{}t)
	\end{gather*}
	График 25.
	
	В этом случае вещественная часть должна быть отрицательна.
	\item
	Только мнимая часть (без вещественной).
	$$y(t)=C_1\sin(\beta{}t)$$
	График 26.
	
	В этом случае система находится на границе устойчивости.
\end{enumerate}

\textbf{Вывод}: для устойчивости системы вещественная часть корня должна быть меньше 0. 

Картинка 27.

Оценивать устойчивость системы решением диффура сложно, поэтому существуют косвенные методы оценки устойчивости, называемые \textbf{критерии устойчивости}.

\subsection{Критерии устойчивости}
Все критерии можно разделить на 2 большие группы:
\begin{itemize}
	\item алгебраические,
	\item часттные.
\end{itemize}

\subsubsection{Алгебраические критерии}
Позволяют судить о устойчивости системы по коэффициентам.

Полином n-го порядка можно заменить n-полиномами 1-го порядка, те $ax^2+bx+c=a(x-x_1)(x-x_2)$. Отсюда следует, что если корни отрицательные, то коэффициенты положительные~--- это \textbf{необходимое условие устойчивости САУ}.

Однако на корни накладываются дополнительные эффекты. Необходимо выявить достаточные условия устойчивости системы. Существуют несколько критериев:
\begin{itemize}
	\item Критерий Раусса,
	\item Критерий Гурвица.
\end{itemize}

\paragraph{Критерий Раусса}
Составляется таблица из коэффициентов. В первой строке записываются четные коэффициенты, во второй нечетные. Каждый следующий элемент таблицы вычисляется по формуле $C_{k,i}=C_{k+1,i-2}-r_iC_{k+1,i-1}$, где $r_i=\frac{C_{1,i-2}}{C_{1,i-1}}$.

\begin{longtable}{*{3}{|c}|}
	&&\\
	\hline
	$a_0$ & $a_2$ & $a_4$\\
	\hline
	$a_1$ & $a_3$ & $a_5$\\
	\hline
\end{longtable}

\begin{gather*}
	C_{1,3}=a_2-\frac{a_0}{a_1}a_3\\
	\ldots
\end{gather*}

Для того, чтобы система была устойчива \textbf{необходимо и достаточно}, чтобы коэффициенты первого столбца были положительными.

\paragraph{Критерий Гурвица}
Составляется матрица по следующему правилу: по главной диагонали слева-направо выписываются все коэффициенты характеристического уравнения, начиная с первого $a_1$. Вверх от главной диагонали пишутся коэффициенты с последовательно возрастающими импульсами, а вниз записываются коэффициенты с последовательно убывающими импульсами. Пустые места заполняются нулями.
$$
\begin{Vmatrix}
	a_1 & a_3 & a_5 & \ldots & 0 \\
	a_0 & a_2 & a_4 & \ldots & 0 \\
	0 & a_1 & a_3 & \ldots & 0 \\
	\hdotsfor{5} \\
	0 & 0 & 0 & \ldots & a_n \\
\end{Vmatrix}
$$
Находятся определители этой матрицы. И критерий звучит так: для того, чтобы система была устойчива \textbf{необходимо и достаточно}, чтобы все определители были больше 0. Те $a_1>0$,$(a_1a_2-a_0a_3)>0$ и т.д.

\subsection{Частотные критерии}
Существует 2 критерия устойчивости:
\begin{itemize}
	\item Критерий Михайлова,
	\item Критерий Найквиста.
\end{itemize}

\subsubsection{Критерий Михайлова}
Основой для анализа является характеристическое уравнение, но в качестве аргумента используется мнимая частота $p=j\omega$. Для определения устойчивости необходимо построить годограф Михайлова.

Линейная система n-го порядка устойчива, если при изменении частоты от 0 до $\infty$ годограф Михайлова последовательно обходит n квадрантов комплексной плоскости против часовой стрелки, начинаясь на положительной части вещественной оси и не проходя через начало координат.

Картинка 28.

Допустим имеется система 1-го порядка, те $D(j\omega)=a_0p+a_1=a_0(p-p_1)=a_0(j\omega-p_1)$. Возьмем частоту $=0$, тогда в скобках остается $-p_1$, который для устойчивости должен быть $<0$, значит все произведение $>0$.

Картинка 29.

Допустим имеется система 2-го порядка, те $D(j\omega)=a_0(j\omega-p_1)(j\omega-p_2)$. Если имеется 2 сомножителя, то фазы складываются, те $\phi_{\Sigma}=\phi_1+\phi_2$.

Если хотя бы один из корней положительный, то годограф не сможет обойти нужное количество квадрантов. Если все корни положительные, то годограф начинается в отрицательной части. Если один из корней не имеет вещественной части, то годограф пройдет через начало координат.

\subsubsection{Критерий Найквиста}
На основе этого критерия можно судить об устойчивости замкнутой системы по виду годографа передаточной функции разомкнутой системы.

Передаточная функция, как правило, записывается в виде отношения полиномов, причем степень числителя, как правило, меньше степени знаменателя. Отсюда можно сделать вывод, что если частота стремится к $\infty$, то дробь и годограф стремятся к 0 (к началу координат).

Поведение годографа в области 0-х частот зависит от вида системы. С точки зрения поведения системы могут быть \textbf{статические и астатические}. Различаются по наличию сомножителя p в знаменателе.

\paragraph{Статические}
Если подставить в это выражение $\omega=0$, то получим точку на положительной части вещественной оси. 

Для определения поведения введем вспомогательную передаточную функцию $W_1(p)=W(p)+1$ или $W_1(j\omega)=W(j\omega)+1=\frac{D(j\omega)}{Q(j\omega)}$. При изменении частоты от 0 до $\infty$ изменение фазы годографа числителя совпадает с изменением фазы годографа знаменателя, те результирующая фаза равна 0. Те при стремлении к $\infty$ фаза годографа стремится к 0. Те годограф не должен охватывать начало координат. Значит годограф исходной функции не должен охватывать точку с координатами $(-1;j0)$.

Картинка 30.

\paragraph{Астатические}
В астатической системе имеется сомножитель $p$. Для определения устойчивости в этомслучае годограф дополняется кривой ... радиуса, начинающегося в положительной части оси $u$ и заканчивающейся в точке $\omega\rightarrow0$

\subsubsection{Логарифмический критерий}
Картинка 33.

Основан на анализе логарифмических характеристик. Система устойчива, если длинна вектора при сдвиге фаз $-\pi$ должна быть меньше 1, что в логарифмических характеристиках значит, что при сдвиге фаз $-\pi$, ЛАЧХ меньше 0.

Картинка 34.

С помощью этих критериев можно определить запас устойчивости по амплитуде и фазе.

\section{Законы управления}
Это алгоритмы или функциональные зависимости, в соответствии с котороми управляющее устройство формирует управляющее воздействие.

Картинка 35.

На практике $u(t)=F_1(x)+F_2(g)+F_3(f)$. Тогда получается, что каждая функция независима от другой и законов получается 3:
\begin{itemize}
	\item Управление (регулирование) по отклонению (первое слогаемое).
	\item Управление по внешнему воздействию (2, 3).
		\begin{itemize}
			\item Управление по задающему воздействию (2),
			\item Управление по возмущающему воздействию (3).
		\end{itemize}
\end{itemize}

Наиболее распространены линейные уравнения, поэтому ограничимся их изучением.

\subsection{Пропорциональное управление}
\begin{align*}
	u(t)&=W_{per}(p)*x(t)\\
	y(t)&=W_0(p)\\
	u(t)&=k_1x(t)\\
	W(p)&=k_1W_0(p)
\end{align*}

Найдем передаточную функцию в установившемся состоянии (по окончании переходных процессов, те производная сигнала равна 0).

Найдем $\lim\limits_{p\rightarrow0}W(p)=k_1k_0=k=\frac{y_{\text{уст}}}{x_0}$. В числителе установившееся значение выходной величины, а в знаменателе постоянное значение ошибки.

Найдем установившуюся ошибку в замкнутой системе. Для этого вспомним уравнение (*).
$$
	x_{\text{уст}}=\frac{g_0}{1+k}-\frac{x_{f\text{уст}}}{1+k}
$$
Где $k$~--- коэффициент усиления, а $g_0$~--- установившееся воздействие, $x_{f\text{уст}}$~--- установившаяся ошибка от коэфициента усиления в этой системе (эта ошибка не равна 0, а уменьшеному в $k+1$ раз входное воздействие).

\subsection{Интегральное управление}
\begin{align*}
	u(t) &= k_2\int{}x(t)dt\\
	u(t) &= \frac{k_2}{p}x(t)\\
	W(p) &= \frac{k_2}pW_0(p)\\
	\lim\limits_{p\rightarrow0}\frac{g_0}{1+W(p)}=0
\end{align*}

Поведение второго слогаемого зависит от поведения числителя.

Таким образом, если имеются интегрирующие звенья, то установившееся значение стремиться к 0.

Картинка 36.

Такая система является астатической по задающему воздействию.

Если задающее воздействие не постоянное, а изменяется с постоянной скоростью (картинка 37). Для избежания ошибки по скорости требуется добавить еще одно интегральное звено, те взять двойной интеграл, тогда система будет астатической второй степени.

Но при увеличении числа интеграторов замедляется процесс управления и изменяется устойчивость (как правило ухудшается).

\subsection{Регулирование по производной}
В состав системы включаются дифференцирующие звенья. Управляющий сигнал будет пропорционален производной сигнала.

Очевидно, что этот закон действует только для изменяющихся значений (производная постоянного значения равна 0). Поэтому данную систему не используют в чистом виде, а только в качестве дополнительных звеньев. 

Плюсы данного управления в скорости.
\begin{align*}
	u(t) &= k_3\frac{dx}{dt}\\
	W_{per}(p) &= k_3p
\end{align*}

\subsection{Изодронное управление}
Сочетает преимущества всех трех предыдущих законов.

Картинка 38.

$$
	u(t) = k_1x(t)+\frac{k_2}{p}x(t)+k_3px(t)
$$

В какой-то момент $t_0$ поступил входной сигнал и началось управление (ошибка изменяются по закону $x=at$). На начальном участке $(0,1)$ интегральная и пропорционнальные части очень малы, поэтому существенный вклад вносит производная $k_3a$. Дальше на участке $(1,2)$ вступает в силу пропорциональный закон(интеграл еще мал, а производная остается постоянной). И дальше вступает в силу интегральный закон.

\section{Коэффициенты ошибок}
Речь идет о нахождении величины ошибки, которая имеет место в процессе управления. Один из способов нахождения ошибок был рассмотрен выше, поэтому рассмотрим иной подход. Ошибки могут быть найдены, как в случае задающего воздействия, так и в случае возмущающего. Найдем изображение ошибки:
$$
	X(p)=\text{Ф}_x(p)G(p)=\frac{G(p)}{1+W(p)}
$$
Разложим передаточную функцию в ряд:
$$
	X(p)=[c_0+c_1p+\frac{c_2}{2!}p^2+\ldots]G(p)
$$
Перейдем к оригиналу:
$$
	x_{\text{уст}}=c_0g(t)+c_1\frac{dg}{dt}+frac{c_2d^2g}{dt^2}+\ldots
$$
Эти $c$~--- и есть коэффициенты ошибок, позволяющие связать входной сигнал с ошибкой. Их можно найти 2я способами: разложением в ряд Тейлора.
\begin{align*}
	c_0=\text{Ф}_x(p)|_{p=0} \\
	c_1=[\frac{d\text{Ф}_x}{dp}]|_{p=0} \\
	\dots
\end{align*}
Эти коэффициенты ошибок принимают различные значения, позволяющие качественно оценить систему. Таким образом для статической системы $c_0\neq0$, для астатической системы 1-го порядка $c_0=0,c_1\neq0$ и т.д.

\paragraph{Пример}
Определить коэффициенты ошибок.
Передаточная функция разомкнутой системы имеет вид:
$$
	W(p)=\frac{k_v}{p(1+T_1p)(1+T_2p)}
$$
Найдем передаточную функцию замкнутой системы:
$$
	\text{Ф}_x(p)=\frac{1}{1+W(p)}=\frac{T_1T_2p^3+(T_1+T_2)p^2+p}{T_1T_2p^3+(T_1+T_2)p^2+p+k_v}
$$
Поделим числитель на знаменатель:
$$
	\text{Ф}_x(p)=\frac1{k_v}p+(\frac{T_1+T_2}{k_v}-\frac1{k_v^2})p^2+(T_1T_2-2\frac{T_1+T_2}{k_v}+\frac1{k_v^2})p^3+\ldots
$$
\begin{align*}
	c_0=0 \\
	c_1=\frac1{k_v} \\
	c_2=\frac{T_1+T_2}{k_v}-\frac1{k_v^2} \\
	\ldots
\end{align*}
Допустим $g(t)=g_0+v_0t+\frac{\epsilon{}t^2}{2}$, найдем установившуюся ошибку:
$$
	x_{\text{уст}}=\frac{v_0+\epsilon{}t}{k_v}+\epsilon\frac{T_1+T_2}{k_v}-\frac1{k_v^2}
$$
Таким образом имеется ошибка по скорости и по ускорению.

\section{Качество переходного процесса, показатели качества}
Требования к качеству процесса управления могут различаться в каждом конкретном случае, но как правило оценивается характер переходного процесса при ступенчатом входном воздействии.

Рисунок 39.

Тут имеются следующие показатели качества:
\begin{itemize}
	\item Время регулирования или длительность переходного процесса ($t_{\text{рег}}$). Это время в течении которого, начиная с момента приложения входного воздействия, отклонение регулируемой величины от ее установившегося значения становится меньше наперед заданного значения $\Delta$. Обычно $\Delta=5\%$.
	\item Перерегулирование ($\sigma$) или заброс. Это максимальное относительное отклонение выходной величины от установившегося значения.
	\item Колебательность. Количество полных колебаний выходной величины за время регулирования.
	\item Установившаяся ошибка. Разность между задающим и выходным воздействием. Не равна 0 только в статических системах.
\end{itemize}

\subsection{Построение переходного процесса}
Для определения показателей качества необходимо иметь график или функцию переходного процесса. Методы получения переходных функций:
\begin{itemize}
	\item Решение диффуравнения.
	\item С использованием передаточных функций.
	\item 
\end{itemize}

\section{Пропуск}

Он используется для определения запаса устойчивости. Находится модуль частотной передаточной функции замкнутой системы $|\text{Ф}(j\omega)|=M$. Рисунок 40.

С точки зрения влияния на устойчивость более опасен график 1, поскольку у него наличиствует резонансный пик. Чем выше резонансный пик, тем сильней колебания (больше склонность системы к колебаниям), тем меньше устойчивость системы. Оценка высоты резонансного пика дает возможность оценить устойчивость системы.
$$
	M_k=\frac{M_{max}}{M_0}
$$
Этот показатель можно оценить на основе передаточной функции разомкнутой системы. При условии, что $M(0)=1$.
\begin{align*}
	M&=|\text{Ф}(j\omega)|=|\frac{W(j\omega}{1+W(j\omega)}| \\
	W(j\omega)&=u(\omega)+jv(\omega) \\
	(u+c)^2&+v^2=R^2\\
	c&=\frac{M^2}{M^2-1}\\
	R&=\frac{M}{M^2-1}
\end{align*}
Это уравнение окружности, причем радиус и центер этой окружности зависят от $M$, величины, которую мы хотим оценить.

Рисунок 41.

По этому графику легко определить максимальное значение модуля на пересечении годографа и заготовки. Считается допустимым, когда показатель коллебательности лежит в диапозоне $1{,}1\ldots1{,}5$.

Этот критерий применяется в задачах проектирования системы, удовлетворяющей условиям. Те рисуется окружность интерисующего радиуса и стараются спроектировать систему, чтобы годограф не заходил внутрь этой окружности.

\subsection{Интегральные оценки качества}
Это такие оценки, которые позволяют одним числом оценить несколько показателей качества переходного процесса.

Рисунок 42. Переходный процесс.

В первую очередь оцениваются длительность переходного процесса и текущую ошибку. Для того чтобы оценить их требуется найти площадь между установившимся значением и $x(t)$.
$$
	J_1=\int^{\infty}_0(x(t)-x_{\text{уст}})^2dt
$$

Эта оценка не очень хороша, потому что большую часть вносит первая левая часть площади и стараясь ее уменьшить мы увеличим скорость нарастания сигнала и большому колебанию.

Поэтому применяют \textbf{улучшеную интегральную оценку}.
$$
	J_k=\int^{\infty}_0(x^2(t)+T^2{}{x'}^2)dt
$$

Здесь учитывается квадрат скорости нарастания, но с определенным поправочным коэффициентом с какой-то постояной $T$. Руководствуясь этой оценкой мы получим, что наши колебания будут стремиться к экспаненте.
\section{Синтез САУ}
Задача синтеза обеспечить требуемое качество процесса управления. Она представляет из себя 2 подхода:
\begin{enumerate}
	\item Цель синтеза достигается за счет изменения параметров существующих устройств/звеньев.
	\item Изменяется структурная схема САУ. Введение дополнительных устройств (корректирующие устройства). Существует 4 типа корректирующих устройств:
		\begin{enumerate}
			\item последовательные,
			\item парралельные
				\begin{enumerate}
					\item собственно паралелльные,
					\item местная обратная связь (чаще всего),
				\end{enumerate}
			\item корректирующие устройства по внешнему воздействию,
			\item неединичная главная обратная связь.
		\end{enumerate}
\end{enumerate}
\subsection{Последовательные корректирующие устройства}
Рисунок 43.

Последовательно подключается еще одно звено. Характеристики звена зависят от выбора инженера и поставленной задачи. Наиболее часто используются:
\begin{itemize}
	\item Пропорциональное.
	\begin{align*}
		W_k(p)=k \\
		A(\omega) = A_0(\omega)k \\
		\phi(\omega)=\phi_0(\omega)
	\end{align*}
	\item Интегрирующее звено. $W_k(p)=\frac{k}p$. Снижается быстродействие, ухудшается устойчивость, управление становится из пропорционального интегральным.
\end{itemize}
\subsubsection{Паралелльное корректирующее устройство}
Рисунок 44.

Распространенный пример~--- введение производной от ошибки.
\begin{align*}
	W_k(p)=Tp \\
	W(p)=W_0(p)(1+W_k(p))=W_0(p)(1+Tp) \\
	A(\omega)=A_0(\omega)\sqrt{1+T^2\omega^2} \\
	\phi(\omega)=\phi_0(\omega)+\arctan{}T\omega
\end{align*}

В данном случае при подании сигнала низкой частоты амплитуда изменится незначительно, а если высокой частоты, то амплитуда резко возрастает. Это \textbf{поднятие высоких частот}. Таким образом это \textbf{повышает быстродействие переходных процессов}. Появляется дополнительный положительный сдвиг фаз.

\subsubsection{Местная обратная связь}
Как известно обратная связь бывает положительной (добавление ко входному сигналу) и отрицательной (вычитаение из входного сигнала выходного).

Также связь бывает гибкая (действует во время переходного процесса) и жесткая (постоянно действующая в одной и той же пропорции ошибка, существует постоянный коэффициент пропорциональности).

\paragraph{Положительная жесткая обратная связь}
Рисунок 45.

Допустим основное звено у нас апереодическое первого порядка, а корректирующим пропорциональное.
\begin{align*}
	W_0(p)=\frac{k}{Tp+1} \\
	W_k(p)=k_k \\
	W(p)=\frac{W_0(p)}{1-W_0(p)W_k(p)}=\frac{k_1}{T_1p+1} \\
	k_1 = \frac{k}{1-kk_k} \\
	T_1=\frac{T}{1-kk_k}
\end{align*}
Таким образом увеличился коэффициент усиления, что хорошо. Но увеличилась постоянная времени, что плохо, тк увеличивается время переходного процесса.

\paragraph{Отрицательная жесткая обратная связь}
Рисунок 46.

Таким образом получим.
\begin{align*}
	W_0(p)=\frac{k}{Tp+1} \\
	W_k(p)=k_k \\
	W(p)=\frac{W_0(p)}{1+W_0(p)W_k(p)}=\frac{k_1}{T_1p+1} \\
	k_1 = \frac{k}{1+kk_k} \\
	T_1=\frac{T}{1+kk_k}
\end{align*}

Уменьшилась постоянная времени звена, что уменьшит время переходного процесса, а значит улучшит качество САУ. Также уменьшился коэффициент усиления, но его можно подкорректировать, как расказано ранее.

Допустим у нас $W_0(p)=\frac{k}p$~--- интегрирующее звено, тогда
\begin{align*}
	k_1=\frac1{k_k} \\
	T_1=\frac1{kk_k}
\end{align*}

Охватив интегратор обратной связью мы получаем апереодическое звено, те не только улучшаем характеристики САУ, но и качественно меняем управление.

\subsubsection{Отрицательная гибкая обратная связь}
Для гибкой обратной связи необходимо использовать дифференцирующее звено. Пусть основное звено у нас колебательное.

\begin{align*}
	W_0(p)=\frac{k}p \\
	W_0(p)=\frac{k}{T^2p^2+2\xi{}Tp+1} \\
	W_k(p)=k_kp	 \\
	W(p)=\frac{k}{T^2p^2+2\xi_1Tp+1} \\
	\xi_1=\xi+\frac{kk_k}{2T}
\end{align*}

\subsection{Корректирующее устройсвто по внешнему воздействию}
Как известно основной принцип управляющего устройства ... по величине ошибки. Однако возможно формирование управляющего воздействия непосредственно из самого входного воздействия. В этом случае в САУ получается 2 контура управления: замкнутый (по ошибке) и разомкнутый. Такое управление называют комбинированным. При таком подходе можно свести величину установившейся ошибки в системе к 0 при любой форме внешнего воздействия. Другими словами система инвариантна к внешнему воздействию.

Возможны 2 ситуации.
\subsubsection{Корректирующее устройство по задающему воздействию}
Рисунок 47.

Найдем передаточную функцию замкнутой системы и передаточную функцию замкнутой системы по ошибке.
\begin{align*}
	\text{Ф}(p)=\frac{W_0(p)}{1+W_0(p)}(1+W_k(p))=\frac{X(p)}{G(p)} \\
	\text{Ф}_{\epsilon}(p)=\frac{\epsilon(p)}{G(p)}=1-\text{Ф}(p)=\frac{1-W_k(p)W_0(p)}{1+W_0(p)}
\end{align*}

В данном случае ни устойчивость ни качесвто переходного процесса не меняются. Необходимо определить условие при котором установившаяся ошибка будет равна 0. Те передаточная функция замкнутой системы должна быть равна 0. Отсюда получаем $W_k(p)=\frac1{W_0(p)}$. Это условие полной инвариантности. В общем случае $W_k(p)=a_0+\tau_1p+\tau^2_2p^2+\ldots$. Таким образом полная инвариантность невозможна или труднодостижима из за бесконечности полученого ряда. Поэтому часто используют частичную инвариантность, например отбрасывают производные высоких порядков предположив, что они равны 0. Также можно исследовать частотный диапозон и выбирать степень производных в зависимости от частоты, добиваясь наибольшего провдоподобия.

\subsubsection{Корректирующее устройство по возмущающему воздействию}
Рисунок 48.

Имеется система, на которую кроме полезного сигнала имеется вредное воздействие. Мы хотим "<укротить его">, добавив корректирующее звено. 
\begin{align*}
	\text{Ф}(p)=\frac{X(p)}{F(p)}=\frac{(W_3-W_kW_1)W_2}{1+W_1W_2}\\
	W_k(p)=\frac{W_3(p)}{W_1(p)}
\end{align*}

Рассуждения аналогичны предыдущему случаю. Если степень числителя меньше знаменателя, то возможно достижение полной инвариантности и наоборот.

\subsubsection{Неединичная главная обратная связь}
Структурная схема показана на рисунке 49.
\begin{align*}
	\text{Ф}(p)=\frac{X(p)}{G(p)}=\frac{W_0(p)}{1+W_0(p)W_k(p)} \\
	W_k(p)=1-\frac1{W_0(p)}=a_0-(\tau_1p+\tau_2^2p+\ldots)
\end{align*}

В такой системе должна быть введена положительная обратная связь по производной. Полная инвариантность не достижима в следствии наличия производных высоких порядков. При таком способе коррекции изменяется характеристическое уравнение (знаменатель передаточной функции). В общем случаи система увеличивает коллебательность и попадает на границу устойчивости.

\subsubsection{Примеры синтеза САУ}
\paragraph{Синтез последовательного корректирующего устройства}
Алгоритм:
\begin{enumerate}
	\item Строятся логарифмические характеристики исходной системы.
	\item На основе показателей качества строится желаемая ЛАЧХ.
	\item Находим разность ЛАЧХ и получаем характеристику корректирующего звена.
	\item Выполняется апроксимация и находится передаточная функция корректирующего звена.
\end{enumerate}
Рисунок 50.

Нет фазовой характеристики!!! Предположим ее такой, чтобы мы могли строить новую ЛАЧХ по качеству. Пусть наше САУ неустойчиво. И с точки зрения качества желательно, чтобы наклон характеристики в области частоты среза равнялся $-20\frac{dB}{dek}$.

Рисуем желаемую ЛАЧХ. Рисунок 50. Дорисовываем наклонную, так, чтобы на частоте среза ЛАЧХ пересекала ось под углом $-40\frac{dB}{dek}$. 

Находим разницу между результатами. Находим передаточную функцию по графику:
$$
	W_k(p)=\frac{(T_1p+1)(T_2p+1)}{(T_3p+1)(T_4p+1)}
$$
\paragraph{Синтез паралелльного корректирующего устройства}
$$
	W(p)=\frac{W_0(p)}{1+W_k(p)W_0(p)}
$$
Предположим, что первое слогаемое в знаменателе много меньше второго и отбросим 1.
\begin{align*}
	W(p)=\frac1{W_k(p)} \\
	L_k(\omega)+L_0(\omega)>>0
\end{align*}

Такимо образом второе условие сильно зависит от частоты, поэтому упрощение возможно только в пределах определенной частоты.
\begin{enumerate}
	\item Строим логарифмические характеристики исходной системы.
	\item На основе заданный критериев качества строим ЛЧХ требуемой системы.
	\item Определяем интервал частот для который выполняются условия.
	\item Определяем логарифмическую характеристику корректирующего устройства.
	\item Аппроксимируем характеристику и находим передаточную функцию корректирующего звена.
\end{enumerate}
$$
	W_k(p)=\frac{kp^2}{T_3p+1}
$$
\section{Разработка системы автоматическогои процессов управления}
Самостоятельно по методичке!!!!

\textbf{Сельсинная пара} состоит из 2х моторчиков, вырабатывающее напряжение из за разности углов двух осей.
\section{Случайные процессы в САУ}
\subsection{Случайные процессы в САУ}
Выше предпологалось, что на вход САУ поступают детерминированные воздействия, однако во многих случаях воздействия носят случайных характер. При оценке таких воздействий можно оперировать только вероятностными характеристиками. 

\textbf{Характеристики случайных величин}
\begin{itemize}
	\item Случайный процесс~--- изменение случайной величины во времени.
	\item Величины бывают непрерывные и дискретные.
	\item Вероятность~--- это предельное значение частоты появления события.
	\item Плотность вероятности~--- это вероятность попадания значения в определенный диапозон.
	\item Математическое ожидание~--- среднее значение случайной величины, с учетом вероятности появления значений.
	\item Дисперсия~--- разброс значений сл. величины от мат. ожидания.
	\item Корреляционная функция~--- степень связаности.
\end{itemize}

\textbf{Законы распределения}: равномерное, нормальное (закон распространения нормально-распределенной СВ при прохождении через САУ не меняется).

Случайные процессы делятся на стационарные и эргодические.


\begin{align*}
	S(\omega)=\int^{\infty}_{\infty}R(\tau)e^{-j\omega{}\tau}d\tau \\
	R(\omega) =\frac1{2\pi}\int^{\infty}_{\infty}S(\omega}e^{j\omega{}\tau}d\tau
\end{align*}
Где $R(\tau)$~--- корреляционная функция, где $\tau$~--- расстояние между отчетами. Корреляционная функция от 0 равняется дисперсии. Если рорелляционная функция длинная, то значит зависимость между элементами сильна и график функции меняет свое значение медленно/плавно.

Рисунок 53.

\subsection{Прохождение случанйого сигнала через линейную непрерывную систему}
Рисунок 54.

Имеется система управления, на вход подается случайный сигнал. \textbf{Общая задача}: нахождение распределения на выходе, зная распределение на входе. \textbf{Частная задача}:найти мат.ожидание, дисперсию, корреляционной функции на выходе, зная мат.ожидание, дисперсию, корреляционной функции на входе. 

Частную задачу решают двумя способами:во временной области и в частотной области(области изображения).

Для нахождения мат.ожидания можно воспользоваться формулой свертки:
$$
	x_2(t)=\int^t_0w(\tau)x_1(t-\tau)d\tau
$$
Это формула для общего случая, но подходит и для мат. ожидания.
$$
	\overline{x_2}(t)=\int^t_0w(\tau)\overline{x_1}(t-\tau)d\tau
$$

Для получения дисперсии необходимо найти корреляционную функцию:
$$
	R_2(t,t_1)=\int^t_0w(\eta)d\eta\int^{t_1}_0w(\lambda)R_1(t-\eta,t_1-\lambda)d\lambda
$$
Если процесс стационарный, то нам не важно когда брать первый отчет и интеграл значительно упрощается:
$$
	R_2(\tau)=R_1(\tau-\eta+\lambda)
$$
Зная корреляционную функцию найдем дисперсию, как $R_2(0)=D_2$

\section{Особые САУ}